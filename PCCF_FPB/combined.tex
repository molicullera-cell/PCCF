%%
% Copyright (c) 2017 - 2023, Pascal Wagler;
% Copyright (c) 2014 - 2023, John MacFarlane
%
% All rights reserved.
%
% Redistribution and use in source and binary forms, with or without
% modification, are permitted provided that the following conditions
% are met:
%
% - Redistributions of source code must retain the above copyright
% notice, this list of conditions and the following disclaimer.
%
% - Redistributions in binary form must reproduce the above copyright
% notice, this list of conditions and the following disclaimer in the
% documentation and/or other materials provided with the distribution.
%
% - Neither the name of John MacFarlane nor the names of other
% contributors may be used to endorse or promote products derived
% from this software without specific prior written permission.
%
% THIS SOFTWARE IS PROVIDED BY THE COPYRIGHT HOLDERS AND CONTRIBUTORS
% "AS IS" AND ANY EXPRESS OR IMPLIED WARRANTIES, INCLUDING, BUT NOT
% LIMITED TO, THE IMPLIED WARRANTIES OF MERCHANTABILITY AND FITNESS
% FOR A PARTICULAR PURPOSE ARE DISCLAIMED. IN NO EVENT SHALL THE
% COPYRIGHT OWNER OR CONTRIBUTORS BE LIABLE FOR ANY DIRECT, INDIRECT,
% INCIDENTAL, SPECIAL, EXEMPLARY, OR CONSEQUENTIAL DAMAGES (INCLUDING,
% BUT NOT LIMITED TO, PROCUREMENT OF SUBSTITUTE GOODS OR SERVICES;
% LOSS OF USE, DATA, OR PROFITS; OR BUSINESS INTERRUPTION) HOWEVER
% CAUSED AND ON ANY THEORY OF LIABILITY, WHETHER IN CONTRACT, STRICT
% LIABILITY, OR TORT (INCLUDING NEGLIGENCE OR OTHERWISE) ARISING IN
% ANY WAY OUT OF THE USE OF THIS SOFTWARE, EVEN IF ADVISED OF THE
% POSSIBILITY OF SUCH DAMAGE.
%%

%%
% This is the Eisvogel pandoc LaTeX template.
%
% For usage information and examples visit the official GitHub page:
% https://github.com/Wandmalfarbe/pandoc-latex-template
%%

% Options for packages loaded elsewhere
\PassOptionsToPackage{unicode}{hyperref}
\PassOptionsToPackage{hyphens}{url}
\PassOptionsToPackage{dvipsnames,svgnames,x11names,table}{xcolor}
%
\documentclass[
  paper=a4,
  ,captions=tableheading
]{scrartcl}
\usepackage{amsmath,amssymb}
% Use setspace anyway because we change the default line spacing.
% The spacing is changed early to affect the titlepage and the TOC.
\usepackage{setspace}
\setstretch{1.2}
\usepackage{iftex}
\ifPDFTeX
  \usepackage[T1]{fontenc}
  \usepackage[utf8]{inputenc}
  \usepackage{textcomp} % provide euro and other symbols
\else % if luatex or xetex
  \usepackage{unicode-math} % this also loads fontspec
  \defaultfontfeatures{Scale=MatchLowercase}
  \defaultfontfeatures[\rmfamily]{Ligatures=TeX,Scale=1}
\fi
\usepackage{lmodern}
\ifPDFTeX\else
  % xetex/luatex font selection
\fi
% Use upquote if available, for straight quotes in verbatim environments
\IfFileExists{upquote.sty}{\usepackage{upquote}}{}
\IfFileExists{microtype.sty}{% use microtype if available
  \usepackage[]{microtype}
  \UseMicrotypeSet[protrusion]{basicmath} % disable protrusion for tt fonts
}{}
\makeatletter
\@ifundefined{KOMAClassName}{% if non-KOMA class
  \IfFileExists{parskip.sty}{%
    \usepackage{parskip}
  }{% else
    \setlength{\parindent}{0pt}
    \setlength{\parskip}{6pt plus 2pt minus 1pt}}
}{% if KOMA class
  \KOMAoptions{parskip=half}}
\makeatother
\usepackage{xcolor}
\definecolor{default-linkcolor}{HTML}{A50000}
\definecolor{default-filecolor}{HTML}{A50000}
\definecolor{default-citecolor}{HTML}{4077C0}
\definecolor{default-urlcolor}{HTML}{4077C0}
\usepackage[left=2cm,right=2cm,top=3cm,bottom=3cm]{geometry}
\usepackage{listings}
\newcommand{\passthrough}[1]{#1}
\lstset{defaultdialect=[5.3]Lua}
\lstset{defaultdialect=[x86masm]Assembler}
\usepackage{longtable,booktabs,array}
\usepackage{calc} % for calculating minipage widths
% Correct order of tables after \paragraph or \subparagraph
\usepackage{etoolbox}
\makeatletter
\patchcmd\longtable{\par}{\if@noskipsec\mbox{}\fi\par}{}{}
\makeatother
% Allow footnotes in longtable head/foot
\IfFileExists{footnotehyper.sty}{\usepackage{footnotehyper}}{\usepackage{footnote}}
\makesavenoteenv{longtable}
% add backlinks to footnote references, cf. https://tex.stackexchange.com/questions/302266/make-footnote-clickable-both-ways
\usepackage{footnotebackref}
\setlength{\emergencystretch}{3em} % prevent overfull lines
\providecommand{\tightlist}{%
  \setlength{\itemsep}{0pt}\setlength{\parskip}{0pt}}
\setcounter{secnumdepth}{5}
\ifLuaTeX
\usepackage[bidi=basic]{babel}
\else
\usepackage[bidi=default]{babel}
\fi
\babelprovide[main,import]{catalan}
% get rid of language-specific shorthands (see #6817):
\let\LanguageShortHands\languageshorthands
\def\languageshorthands#1{}
% 📦 Paquets bàsics
\usepackage{array}
\usepackage{graphicx}
\graphicspath{{docs/img/}{docs/img/logos/}}
\usepackage{fontspec}
\usepackage{titling}
\usepackage{lastpage}
\usepackage{longtable}
\usepackage{booktabs}
\usepackage{xcolor,colortbl}
\usepackage{geometry}
\geometry{head=28pt}
\usepackage{csquotes}
\usepackage{xltxtra}
\usepackage{listings}
\usepackage{awesomebox, tcolorbox}
\usepackage{tabularx}
\newcolumntype{Y}{>{\raggedright\arraybackslash}X}
\usepackage[table]{xcolor}
\definecolor{HTML970097}{HTML}{970097}


% 🎨 Colors personalitzats
\definecolor{morat}{rgb}{0.396, 0.188, 0.592}
\definecolor{groc}{rgb}{0.984, 0.612, 0.031}
\definecolor{lightblue}{rgb}{0.68, 0.85, 0.9}
\definecolor{ballblue}{rgb}{0.13, 0.67, 0.8}
\definecolor{cerulean}{rgb}{0.0, 0.48, 0.65}
\definecolor{almond}{rgb}{0.94, 0.87, 0.8}
\definecolor{apricot}{rgb}{0.98, 0.81, 0.69}
\definecolor{cream}{rgb}{1.0, 0.99, 0.82}
\definecolor{coralred}{rgb}{1.0, 0.25, 0.25}
\definecolor{headerblue}{RGB}{26,72,110}
\definecolor{headergreen}{RGB}{3,75,33}
\definecolor{headerpurple}{RGB}{102,58,143}
\definecolor{rowgray}{RGB}{240,240,240}
\definecolor{bordergray}{RGB}{200,200,200}
\definecolor{cellgreen}{RGB}{226,239,218}
\definecolor{cellblue}{RGB}{198,217,241}

% 📑 Estils seccions
\usepackage{sectsty}
\addtokomafont{section}{\color{morat}}
\addtokomafont{subsection}{\color{morat}}
\addtokomafont{subsubsection}{\color{morat}}

% 📊 Estils taula
\renewcommand{\arraystretch}{1.3}
\setlength{\arrayrulewidth}{1pt}
\arrayrulecolor{bordergray}
\setlength{\tabcolsep}{8pt}
\newcolumntype{L}[1]{>{\raggedright\arraybackslash}p{#1}}
\newcolumntype{C}[1]{>{\centering\arraybackslash}p{#1}}
\newcolumntype{R}[1]{>{\raggedleft\arraybackslash}p{#1}}
\AtBeginEnvironment{longtable}{\rowcolors{2}{rowgray}{white}}
\AtBeginEnvironment{tabular}{\rowcolors{2}{rowgray}{white}}

\newcommand{\tableheaderstyle}{\rowcolor{headerblue}\color{white}\bfseries}
\newcommand{\sectionheaderstyle}{\rowcolor{headergreen}\color{white}\bfseries}
\newcommand{\contentheaderstyle}{\rowcolor{headerpurple}\color{white}\bfseries}
\newcommand{\saberfer}{\cellcolor{cellgreen}}
\newcommand{\saberestar}{\cellcolor{cellblue}}

\ifLuaTeX
  \usepackage{selnolig}  % disable illegal ligatures
\fi
\IfFileExists{bookmark.sty}{\usepackage{bookmark}}{\usepackage{hyperref}}
\IfFileExists{xurl.sty}{\usepackage{xurl}}{} % add URL line breaks if available
\urlstyle{same}
\hypersetup{
  pdftitle={Proposta curricular de cicle formatiu FPB Informàtica d'Oficina},
  pdfauthor={JB Talens},
  pdflang={ca},
  hidelinks,
  breaklinks=true,
  pdfcreator={LaTeX via pandoc with the Eisvogel template}}
\title{Proposta curricular de cicle formatiu \newline FPB Informàtica
d'Oficina}
\usepackage{etoolbox}
\makeatletter
\providecommand{\subtitle}[1]{% add subtitle to \maketitle
  \apptocmd{\@title}{\par {\large #1 \par}}{}{}
}
\makeatother
\subtitle{FPB Informàtica i Comunicacions \newline \today}
\author{JB Talens}
\date{}



%%
%% added
%%

\usepackage[pages=all]{background}

%
% for the background color of the title page
%
\usepackage{pagecolor}
\usepackage{afterpage}
\usepackage{tikz}

%
% break urls
%
\PassOptionsToPackage{hyphens}{url}

%
% When using babel or polyglossia with biblatex, loading csquotes is recommended
% to ensure that quoted texts are typeset according to the rules of your main language.
%
\usepackage{csquotes}

%
% captions
%
\definecolor{caption-color}{HTML}{777777}
\usepackage[font={stretch=1.2}, textfont={color=caption-color}, position=top, skip=4mm, labelfont=bf, singlelinecheck=false, justification=raggedright]{caption}
\setcapindent{0em}

%
% blockquote
%
\definecolor{blockquote-border}{RGB}{221,221,221}
\definecolor{blockquote-text}{RGB}{119,119,119}
\usepackage{mdframed}
\newmdenv[rightline=false,bottomline=false,topline=false,linewidth=3pt,linecolor=blockquote-border,skipabove=\parskip]{customblockquote}
\renewenvironment{quote}{\begin{customblockquote}\list{}{\rightmargin=0em\leftmargin=0em}%
\item\relax\color{blockquote-text}\ignorespaces}{\unskip\unskip\endlist\end{customblockquote}}

%
% Source Sans Pro as the default font family
% Source Code Pro for monospace text
%
% 'default' option sets the default
% font family to Source Sans Pro, not \sfdefault.
%
\ifnum 0\ifxetex 1\fi\ifluatex 1\fi=0 % if pdftex
    \usepackage[default]{sourcesanspro}
  \usepackage{sourcecodepro}
  \else % if not pdftex
    \usepackage[default]{sourcesanspro}
  \usepackage{sourcecodepro}

  % XeLaTeX specific adjustments for straight quotes: https://tex.stackexchange.com/a/354887
  % This issue is already fixed (see https://github.com/silkeh/latex-sourcecodepro/pull/5) but the
  % fix is still unreleased.
  % TODO: Remove this workaround when the new version of sourcecodepro is released on CTAN.
  \ifxetex
    \makeatletter
    \defaultfontfeatures[\ttfamily]
      { Numbers   = \sourcecodepro@figurestyle,
        Scale     = \SourceCodePro@scale,
        Extension = .otf }
    \setmonofont
      [ UprightFont    = *-\sourcecodepro@regstyle,
        ItalicFont     = *-\sourcecodepro@regstyle It,
        BoldFont       = *-\sourcecodepro@boldstyle,
        BoldItalicFont = *-\sourcecodepro@boldstyle It ]
      {SourceCodePro}
    \makeatother
  \fi
  \fi

%
% heading color
%
\definecolor{heading-color}{RGB}{40,40,40}
\addtokomafont{section}{\color{heading-color}}
% When using the classes report, scrreprt, book,
% scrbook or memoir, uncomment the following line.
%\addtokomafont{chapter}{\color{heading-color}}

%
% variables for title, author and date
%
\usepackage{titling}
\title{Proposta curricular de cicle formatiu \newline FPB Informàtica
d'Oficina}
\author{JB Talens}
\date{}

%
% tables
%

\definecolor{table-row-color}{HTML}{F5F5F5}
\definecolor{table-rule-color}{HTML}{999999}

%\arrayrulecolor{black!40}
\arrayrulecolor{table-rule-color}     % color of \toprule, \midrule, \bottomrule
\setlength\heavyrulewidth{0.3ex}      % thickness of \toprule, \bottomrule
\renewcommand{\arraystretch}{1.3}     % spacing (padding)


%
% remove paragraph indentation
%
\setlength{\parindent}{0pt}
\setlength{\parskip}{6pt plus 2pt minus 1pt}
\setlength{\emergencystretch}{3em}  % prevent overfull lines

%
%
% Listings
%
%


%
% general listing colors
%
\definecolor{listing-background}{HTML}{F7F7F7}
\definecolor{listing-rule}{HTML}{B3B2B3}
\definecolor{listing-numbers}{HTML}{B3B2B3}
\definecolor{listing-text-color}{HTML}{000000}
\definecolor{listing-keyword}{HTML}{435489}
\definecolor{listing-keyword-2}{HTML}{1284CA} % additional keywords
\definecolor{listing-keyword-3}{HTML}{9137CB} % additional keywords
\definecolor{listing-identifier}{HTML}{435489}
\definecolor{listing-string}{HTML}{00999A}
\definecolor{listing-comment}{HTML}{8E8E8E}

\lstdefinestyle{eisvogel_listing_style}{
  language         = java,
  numbers          = left,
  xleftmargin      = 2.7em,
  framexleftmargin = 2.5em,
  backgroundcolor  = \color{listing-background},
  basicstyle       = \color{listing-text-color}\linespread{1.0}%
                      \lst@ifdisplaystyle%
                      \small%
                      \fi\ttfamily{},
  breaklines       = true,
  frame            = single,
  framesep         = 0.19em,
  rulecolor        = \color{listing-rule},
  frameround       = ffff,
  tabsize          = 4,
  numberstyle      = \color{listing-numbers},
  aboveskip        = 1.0em,
  belowskip        = 0.1em,
  abovecaptionskip = 0em,
  belowcaptionskip = 1.0em,
  keywordstyle     = {\color{listing-keyword}\bfseries},
  keywordstyle     = {[2]\color{listing-keyword-2}\bfseries},
  keywordstyle     = {[3]\color{listing-keyword-3}\bfseries\itshape},
  sensitive        = true,
  identifierstyle  = \color{listing-identifier},
  commentstyle     = \color{listing-comment},
  stringstyle      = \color{listing-string},
  showstringspaces = false,
  escapeinside     = {/*@}{@*/}, % Allow LaTeX inside these special comments
  literate         =
  {á}{{\'a}}1 {é}{{\'e}}1 {í}{{\'i}}1 {ó}{{\'o}}1 {ú}{{\'u}}1
  {Á}{{\'A}}1 {É}{{\'E}}1 {Í}{{\'I}}1 {Ó}{{\'O}}1 {Ú}{{\'U}}1
  {à}{{\`a}}1 {è}{{\`e}}1 {ì}{{\`i}}1 {ò}{{\`o}}1 {ù}{{\`u}}1
  {À}{{\`A}}1 {È}{{\`E}}1 {Ì}{{\`I}}1 {Ò}{{\`O}}1 {Ù}{{\`U}}1
  {ä}{{\"a}}1 {ë}{{\"e}}1 {ï}{{\"i}}1 {ö}{{\"o}}1 {ü}{{\"u}}1
  {Ä}{{\"A}}1 {Ë}{{\"E}}1 {Ï}{{\"I}}1 {Ö}{{\"O}}1 {Ü}{{\"U}}1
  {â}{{\^a}}1 {ê}{{\^e}}1 {î}{{\^i}}1 {ô}{{\^o}}1 {û}{{\^u}}1
  {Â}{{\^A}}1 {Ê}{{\^E}}1 {Î}{{\^I}}1 {Ô}{{\^O}}1 {Û}{{\^U}}1
  {œ}{{\oe}}1 {Œ}{{\OE}}1 {æ}{{\ae}}1 {Æ}{{\AE}}1 {ß}{{\ss}}1
  {ç}{{\c c}}1 {Ç}{{\c C}}1 {ø}{{\o}}1 {å}{{\r a}}1 {Å}{{\r A}}1
  {€}{{\EUR}}1 {£}{{\pounds}}1 {«}{{\guillemotleft}}1
  {»}{{\guillemotright}}1 {ñ}{{\~n}}1 {Ñ}{{\~N}}1 {¿}{{?`}}1
  {…}{{\ldots}}1 {≥}{{>=}}1 {≤}{{<=}}1 {„}{{\glqq}}1 {“}{{\grqq}}1
  {”}{{''}}1
}
\lstset{style=eisvogel_listing_style}

%
% Java (Java SE 12, 2019-06-22)
%
\lstdefinelanguage{Java}{
  morekeywords={
    % normal keywords (without data types)
    abstract,assert,break,case,catch,class,continue,default,
    do,else,enum,exports,extends,final,finally,for,if,implements,
    import,instanceof,interface,module,native,new,package,private,
    protected,public,requires,return,static,strictfp,super,switch,
    synchronized,this,throw,throws,transient,try,volatile,while,
    % var is an identifier
    var
  },
  morekeywords={[2] % data types
    % primitive data types
    boolean,byte,char,double,float,int,long,short,
    % String
    String,
    % primitive wrapper types
    Boolean,Byte,Character,Double,Float,Integer,Long,Short
    % number types
    Number,AtomicInteger,AtomicLong,BigDecimal,BigInteger,DoubleAccumulator,DoubleAdder,LongAccumulator,LongAdder,Short,
    % other
    Object,Void,void
  },
  morekeywords={[3] % literals
    % reserved words for literal values
    null,true,false,
  },
  sensitive,
  morecomment  = [l]//,
  morecomment  = [s]{/*}{*/},
  morecomment  = [s]{/**}{*/},
  morestring   = [b]",
  morestring   = [b]',
}

\lstdefinelanguage{XML}{
  morestring      = [b]",
  moredelim       = [s][\bfseries\color{listing-keyword}]{<}{\ },
  moredelim       = [s][\bfseries\color{listing-keyword}]{</}{>},
  moredelim       = [l][\bfseries\color{listing-keyword}]{/>},
  moredelim       = [l][\bfseries\color{listing-keyword}]{>},
  morecomment     = [s]{<?}{?>},
  morecomment     = [s]{<!--}{-->},
  commentstyle    = \color{listing-comment},
  stringstyle     = \color{listing-string},
  identifierstyle = \color{listing-identifier}
}

%
% header and footer
%
\usepackage[headsepline,footsepline]{scrlayer-scrpage}

\newpairofpagestyles{eisvogel-header-footer}{
  \clearpairofpagestyles
  \ihead*{\includegraphics[scale=0.5]{docs/img/logos/logoJust.png}
\textcolor{morat}{| Programació d'Aula IMX}}
  \chead*{}
  \ohead*{\includegraphics[scale=0.5]{docs/img/logos/GvaNext.png}}
  \ifoot*{\includegraphics[scale=0.1]{docs/img/logos/FonsVal.png}}
  \cfoot*{\thepage/\pageref{LastPage}}
  \ofoot*{\includegraphics[scale=0.3]{docs/img/logos/miniPla.png}}
  \addtokomafont{pageheadfoot}{\upshape}
}
\pagestyle{eisvogel-header-footer}


\backgroundsetup{
scale=1,
color=black,
opacity=0.2,
angle=0,
contents={%
  \includegraphics[width=\paperwidth,height=\paperheight]{docs/img/fondo.png}
  }%
}

%%
%% end added
%%

\begin{document}

%%
%% begin titlepage
%%
\begin{titlepage}
\newgeometry{top=2cm, right=4cm, bottom=3cm, left=4cm}
\tikz[remember picture,overlay] \node[inner sep=0pt] at (current page.center){\includegraphics[width=\paperwidth,height=\paperheight]{docs/img/portada-aula.png}};
\newcommand{\colorRule}[3][black]{\textcolor[HTML]{#1}{\rule{#2}{#3}}}
\begin{flushleft}
\noindent
\\[-1em]
\color[HTML]{F08A2A}
\makebox[0pt][l]{\colorRule[435488]{1.3\textwidth}{0pt}}
\par
\noindent

% The titlepage with a background image has other text spacing and text size
{
  \setstretch{2}
  \vfill
  \vskip -8em
  \noindent {\huge \textbf{\textsf{Proposta curricular de cicle formatiu
\newline FPB Informàtica d'Oficina}}}
    \vskip 1em
  {\Large \textsf{FPB Informàtica i Comunicacions \newline \today}}
    \vskip 2em
  \noindent {\Large \textsf{JB Talens} \vskip 0.6em \textsf{}}
  \vfill
}


\end{flushleft}
\end{titlepage}
\restoregeometry
\pagenumbering{arabic}

%%
%% end titlepage
%%

% \maketitle


\renewcommand*\contentsname{Índex}
{
\setcounter{tocdepth}{3}
\tableofcontents
\newpage
}
El \emph{Decret 114/2025, de 29 de juliol del Consell, pel qual
s/'establixen els currículums dels cicles formatius de grau mitjà i de
grau superior de Formació Professional}, estableix que cada centre
docent desenvoluparà i completarà, en el marc de la seua autonomia, els
curriculums de cada cicle formatiu que impartisca mitjançant
l/'elaboració de projectes curriculars (PCCF), els quals suposaran el
marc de referència per a la planificació, el desenvolupament i
l/'avaluació del procés d/'ensenyament i aprenentatge del cicle.

El present document, en virtut d/'aquest Decret, desenvolupa i completa
el currículum per al \textbf{CFGM de Sistemes Microinformàtics i
Xarxes}, i l/'adapta a l/'entorn socioeconòmic i empresarial de l/'IES
Jaume II el Just, a Tavernes de la Valldigna. Les programacions
didàctiques de cada mòdul professional s/'elaboraran tenint en compte
aquest projecte, assegurant la coherència perdagògica i metodològica en
la impartició del cicle.

\hypertarget{identificaciuxf3-del-cicle-formatiu}{%
\section{1. Identificació del Cicle
Formatiu}\label{identificaciuxf3-del-cicle-formatiu}}

El primer requeriment d'un projecte curricular de cicle formatiu (PCCF)
és la seua identificació clara. En aquest apartat es recullen les dades
essencials que permeten reconéixer oficialment el cicle objecte de
programació, així com la vinculació amb el centre educatiu i l'equip
docent responsable de la seua implementació. Aquesta informació assegura
la traçabilitat institucional del projecte i el seu encaix amb l'oferta
educativa del centre i del sistema valencià de Formació Professional.

\hypertarget{denominaciuxf3-oficial-del-tuxedtol}{%
\subsection{Denominació oficial del
títol}\label{denominaciuxf3-oficial-del-tuxedtol}}

\begin{itemize}
\tightlist
\item
  Denominació: Títol Professional Bàsic en Informàtica d'Oficina.
\item
  Nivell: Formació Professional Bàsica.
\item
  Durada: 2.000 hores (dos cursos acadèmics).
\item
  Família professional: Informàtica i Comunicacions.
\item
  Normativa estatal de referència: Reial decret 356/2014, de 16 de maig.
\item
  Normativa autonòmica: adaptació curricular segons normativa de la
  Comunitat Valenciana.
\item
  Referència internacional: CINE-3.5.4 (Classificació Internacional
  Normalitzada de l'Educació, UNESCO).
\item
  Marc Espanyol de Qualificacions per a l/'Educació i la Formació
  (MECU): Nivell 1.
\end{itemize}

\hypertarget{centre-educatiu}{%
\subsection{Centre educatiu}\label{centre-educatiu}}

\begin{itemize}
\tightlist
\item
  Nom del centre: IES Jaume II el Just.
\item
  Codi del centre: 46008340.
\item
  Localitat: Tavernes de la Valldigna (La Safor).
\item
  Província: València.
\item
  Tipus de centre: Institut públic d'Educació Secundària, Batxillerat i
  FP.
\item
  Modalitat d'ensenyament: presencial.
\end{itemize}

\hypertarget{equip-educatiu-responsable}{%
\subsection{Equip educatiu
responsable}\label{equip-educatiu-responsable}}

El projecte curricular és responsabilitat de l'equip docent del cicle,
liderat pel Departament d'Informàtica i Comunicacions. Aquest equip
incorpora professorat de mòduls específics i transversals, així com la
figura de la tutoria i la coordinació de Formació Professional Bàsica.
La direcció d'estudis de FP garanteix la integració del projecte en la
programació general anual del centre.

Per al curs 2025-2026, l/'organització docent del cicle incorpora les
següents assignacions específiques:

\begin{itemize}
\tightlist
\item
  El Departament de Castellà assumeix la docència del mòdul transversal
  Comunicació i Societat, excepte els resultats d'aprenentatge vinculats
  a la llengua anglesa.
\item
  El Departament d'Anglés imparteix els resultats d'aprenentatge
  corresponents a la competència lingüística en llengua estrangera
  integrats en el mateix mòdul.
\item
  El Departament de Física i Química es fa càrrec de la docència del
  mòdul Ciències Aplicades.
\end{itemize}

Aquesta distribució respon a criteris d'especialització docent i
d'adaptació curricular que garanteixen una atenció adequada a les
competències bàsiques i professionals requerides

\hypertarget{marc-normatiu-per-al-desplegament-del-projecte-curricular}{%
\section{2. Marc normatiu per al desplegament del projecte
curricular}\label{marc-normatiu-per-al-desplegament-del-projecte-curricular}}

L'elaboració d'aquest \textbf{Projecte Curricular de Cicle Formatiu
(PCCF)} es fonamenta en el seu reconeixement com a instrument de
\textbf{nivell curricular 2}, segons s'estableix en la guia elaborada
per la Direcció General de Formació Professional. En aquest marc, el
PCCF \textbf{no depén de les programacions didàctiques}, sinó que
\textbf{les integra i articula} a través dels \textbf{acords pedagògics,
metodològics i organitzatius} definits col·lectivament per l'equip
docent. Així, esdevé el \textbf{document de referència per a la
planificació i coherència global del cicle}, i orienta el
desenvolupament de les \textbf{programacions d'aula dels mòduls
professionals} (nivell curricular 3).

Tot i aquesta estructuració clara en dos nivells curriculars, cal
destacar que, mentre que en altres etapes educatives com
\textbf{Primària, ESO i Batxillerat} la normativa ha substituït el
concepte de \emph{programació didàctica} pel de \emph{projecte
curricular d'etapa o de cicle}, en l'àmbit de la Formació Professional
\textbf{aquesta substitució terminològica no s'ha produït encara}. Per
tant, en FP es manté l'ús del terme \emph{programació didàctica}, tot i
que el \textbf{Projecte Curricular de Cicle Formatiu (PCCF)}
\textbf{assumeix i integra} tot allò corresponent al \textbf{nivell
curricular 2}, segons les directrius de la Direcció General de Formació
Professional.

Així, entenem que \textbf{el PCCF conté els acords i orientacions
pedagògiques comunes del cicle}, que donen coherència a la pràctica
docent i que \textbf{substitueixen la necessitat de programacions
didàctiques individuals desvinculades}. Per tant, mentre la normativa no
indique el contrari, el PCCF \textbf{integra funcionalment les
programacions didàctiques}, i cada docent ha de \textbf{desenvolupar la
seua pròpia programació d'aula} (nivell curricular 3) \textbf{basant-se
en el marc establit pel PCCF}, coherent amb els criteris d'avaluació,
els resultats d'aprenentatge i les metodologies comunes acordades per
l'equip docent.

D'aquesta manera, es garanteix una coordinació acurada entre els
documents, es reforça la coherència de la pràctica docent i es preserva
el dret de l'alumnat a una avaluació objectiva, personalitzada i
adaptada al seu progrés i circumstàncies, tal com estableix
l'\textbf{Ordre 2025/13083}.

A més, aquesta normativa dona suport a la \textbf{implantació de
metodologies actives}, la \textbf{flexibilitat curricular}, i
l'enfocament \textbf{intermodular}, amb especial atenció a la
\textbf{singularització del projecte per a cada estudiant}. Aquest
plantejament reforça la funció del PCCF com a peça clau per garantir
l'alineació metodològica, la contextualització i l'assoliment dels
resultats d'aprenentatge (RA).

Dins aquest nou escenari, la \textbf{Resolució de 17 de juliol de 2025},
de la Secretaria Autonòmica d'Educació, dicta les \textbf{instruccions
per a l'organització i funcionament dels cicles formatius de grau D i E}
i estableix que el \textbf{PCCF ha d'incloure-se a la PGA} i
\textbf{avaluar-se} com a part essencial de la gestió pedagògica anual.
A més, aquest document ha de reflectir les directrius de centre,
integrar les mesures d'atenció a la diversitat, i establir mecanismes de
seguiment i millora contínua (apartat 22.1 de la resolució).

Finalment, cal tenir en compte que, malgrat la vigència de la nova
\textbf{Llei orgànica 3/2022, de 31 de març}, i del desplegament del
\textbf{Reial decret 659/2023}, determinats aspectes de l'ordenació de
la FPB continuen regulats per normatives anteriors, com ara el
\textbf{Reial decret 127/2014} o el \textbf{Reial decret 356/2014}, tal
com estableixen les disposicions transitòries i el \textbf{calendari
d'implantació del Reial decret 278/2023}. Aquest escenari de transició
normativa implica que els centres han de gestionar una convivència
ordenada entre marcs nous i vigents, cosa que reforça encara més la
funció estructuradora del PCCF com a vertebrador de l'oferta formativa
del cicle.

\hypertarget{normativa-general}{%
\subsection{Normativa general}\label{normativa-general}}

\begin{itemize}
\tightlist
\item
  \textbf{Llei orgànica 2/2006, de 3 de maig}, d'Educació (LOE),
  modificada per:
\item
  \textbf{Llei orgànica 8/2013, de 9 de desembre} (LOMCE)
\item
  \textbf{Llei orgànica 3/2020, de 29 de desembre} (LOMLOE)
\item
  \textbf{Llei orgànica 3/2022, de 31 de març}, d'ordenació i integració
  de la Formació Professional
\item
  \textbf{Reial decret 659/2023, de 18 de juliol}, pel qual es desplega
  l'ordenació del Sistema de Formació Professional
\item
  \textbf{Resolució de 17 de juliol de 2025}, de la Secretaria
  Autonòmica d'Educació, sobre instruccions per a l'FP
\end{itemize}

\hypertarget{normativa-especuxedfica-per-al-tuxedtol}{%
\subsection{Normativa específica per al
títol}\label{normativa-especuxedfica-per-al-tuxedtol}}

\begin{itemize}
\tightlist
\item
  \textbf{Reial decret 356/2014, de 16 de maig}, pel qual s'estableix el
  títol professional bàsic en Informàtica d'Oficina
\item
  \textbf{Decret 185/2014, de 31 d'octubre}, pel qual s'establix el
  currículum del cicle a la Comunitat Valenciana
\end{itemize}

\hypertarget{avaluaciuxf3-i-qualificaciuxf3}{%
\subsection{Avaluació i
qualificació}\label{avaluaciuxf3-i-qualificaciuxf3}}

\begin{itemize}
\tightlist
\item
  \textbf{Ordre 79/2010, de 27 d'agost}, sobre l'avaluació de l'alumnat
  dels cicles formatius a la Comunitat Valenciana
\item
  \textbf{Ordre 2025/13083, de 30 d'abril}, sobre el nou sistema
  d'avaluació en l'FP, amb efectes \textbf{retroactius} des de l'1 de
  setembre de 2024
\item
  \textbf{Reial decret 498/2024, de 21 de maig}, que actualitza els
  criteris d'avaluació i el sistema de qualificacions en cicles de grau
  bàsic
\end{itemize}

\hypertarget{altres-referuxe8ncies-normatives}{%
\subsection{Altres referències
normatives}\label{altres-referuxe8ncies-normatives}}

\begin{itemize}
\tightlist
\item
  \textbf{Reial decret 217/2022}, d'ordenació de l'ESO (en allò
  aplicable als cicles FPB)
\item
  \textbf{Decret 107/2022}, d'ordenació del currículum de l'ESO a la
  Comunitat Valenciana
\item
  \textbf{Reial decret 278/2023}, sobre calendari d'implantació de la
  nova ordenació de l'FP
\item
  \textbf{Resolucions anuals d'inici de curs} de la Conselleria
  d'Educació
\item
  \textbf{Projecte educatiu del centre (PEC)} i \textbf{programació
  general anual (PGA)}
\end{itemize}

Aquest conjunt normatiu dona \textbf{suport i legitimitat a les
decisions recollides en el PCCF}, que esdevé el document estratègic per
\textbf{alinear, coordinar i millorar la pràctica docent} del cicle,
afavorint l'eficàcia pedagògica i l'atenció a la diversitat dins d'un
marc legal en evolució.

\hypertarget{adequaciuxf3-i-adaptaciuxf3-de-les-competuxe8ncies-professionals-del-tuxedtol-al-context-socioeconuxf2mic-i-cultural-del-centre}{%
\section{3. Adequació i adaptació de les competències professionals del
títol al context socioeconòmic i cultural del
centre}\label{adequaciuxf3-i-adaptaciuxf3-de-les-competuxe8ncies-professionals-del-tuxedtol-al-context-socioeconuxf2mic-i-cultural-del-centre}}

L'adaptació de les competències professionals del títol ha de partir
d'una anàlisi del \textbf{context socioeconòmic i cultural del centre
educatiu}, així com de les característiques específiques de l'alumnat.
Aquesta adequació no implica modificar ni suprimir competències
recollides en el perfil professional, sinó \textbf{ponderar-les i
contextualitzar-les} d'acord amb les necessitats del territori i les
orientacions del \textbf{Projecte d'Acció Curricular (PAC)}.

L'\textbf{IES Jaume II el Just}, situat a Tavernes de la Valldigna, es
troba en una posició estratègica per donar resposta a les necessitats
del \textbf{sector TIC} de les comarques de la Safor i la Ribera. El
centre compta amb una àmplia trajectòria dins la \textbf{Família
Professional d'Informàtica i Comunicacions}, i ofereix cicles formatius
de grau bàsic, mitjà i superior, així com \textbf{cursos
d'especialització en Ciberseguretat i Intel·ligència Artificial i Big
Data}.

\hypertarget{anuxe0lisi-del-context-territorial-i-productiu}{%
\subsection{3.1. Anàlisi del context territorial i
productiu}\label{anuxe0lisi-del-context-territorial-i-productiu}}

Segons dades de la Cambra de Comerç, el País Valencià és un dels
principals pols tecnològics de l'Estat espanyol, amb més de
\textbf{2.000 empreses TIC}, de les quals un 70\% són \textbf{pimes}.
Aquest teixit empresarial, amb escassos recursos per a formació pròpia,
demanda professionals polivalents i adaptables.

Les comarques de \textbf{la Safor} i \textbf{la Ribera} han experimentat
un creixement notable del sector TIC, vinculat a la digitalització de
processos empresarials i a la creació d'espais d'innovació, especialment
en àmbits com:

\begin{itemize}
\tightlist
\item
  Digitalització del sector agroalimentari
\item
  Automatització i gestió empresarial
\item
  Suport i manteniment de sistemes i infraestructures TIC
\item
  Desenvolupament d'aplicacions i serveis multiplataforma
\end{itemize}

L'impacte de la \textbf{DANA d'octubre de 2024} ha incrementat, a més,
la necessitat de perfils relacionats amb la \textbf{recuperació de
dades, manteniment i seguretat informàtica}, i la transformació digital
del teixit comercial.

\hypertarget{marc-normatiu-de-referuxe8ncia}{%
\subsection{3.2. Marc normatiu de
referència}\label{marc-normatiu-de-referuxe8ncia}}

Segons el \textbf{Reial decret 498/2024}, que actualitza els perfils
professionals dels cicles de grau bàsic, les antigues /``competències
professionals, personals i socials/'' passen a denominar-se
\textbf{competències professionals i per a l'ocupabilitat}, posant
èmfasi en:

\begin{itemize}
\tightlist
\item
  L'adaptació al canvi tecnològic
\item
  La capacitat de treball en equip
\item
  L'autonomia i la iniciativa professional
\item
  L'assoliment de tasques pròximes a l'entorn real de treball
\end{itemize}

Aquest enfocament està recollit també al \textbf{Decret 185/2014, de 31
d'octubre}, que fixa el currículum del \textbf{Títol Professional Bàsic
en Informàtica d'Oficina} per a la Comunitat Valenciana, vigent fins a
la plena implementació dels nous currículums previstos per la
\textbf{LOFP 3/2022} i el \textbf{RD 659/2023}.

\hypertarget{adequaciuxf3-de-les-competuxe8ncies-al-perfil-territorial}{%
\subsection{3.3. Adequació de les competències al perfil
territorial}\label{adequaciuxf3-de-les-competuxe8ncies-al-perfil-territorial}}

A partir de l'anàlisi del context i la consulta amb els agents
socioeconòmics i empresarials de la zona, s'han identificat com a
\textbf{competències prioritàries} per afavorir la inserció laboral i la
projecció professional dels titulats les següents:

\begin{itemize}
\tightlist
\item
  Manteniment de sistemes microinformàtics i xarxes locals
\item
  Recuperació de dades i diagnòstic de fallades
\item
  Instal·lació i configuració de programari bàsic i d'ofimàtica
\item
  Comunicació professional oral i escrita
\item
  Atenció al client i suport tècnic bàsic
\item
  Gestió de la seguretat i la protecció ambiental
\item
  Autonomia, responsabilitat i treball en equip
\end{itemize}

Aquestes prioritats s'han traslladat al \textbf{desenvolupament
metodològic i curricular} del present PCCF, tot mantenint el marc
competencial oficial però ajustant el \textbf{pes relatiu de cadascuna
de les competències} segons el seu impacte formatiu i ocupacional.

\hypertarget{proposta-de-ponderaciuxf3-de-competuxe8ncies}{%
\subsection{3.4. Proposta de ponderació de
competències}\label{proposta-de-ponderaciuxf3-de-competuxe8ncies}}

Per tal de garantir la coherència entre el perfil del títol i les
demandes de l'entorn productiu, l'equip docent ha consensuat la següent
\textbf{ponderació orientativa} de les competències professionals:

\begin{longtable}[]{@{}lll@{}}
\toprule
\textbf{Codi} & \textbf{Competència professional} & \textbf{Ponderació
(\%)} \\
\midrule
\endhead
a & Preparar equips i aplicacions per al tractament, impressió i arxiu
de dades i textos & 10 \% \\
b & Elaborar documents amb processadors de text i fulls de càlcul
seguint protocols & 10 \% \\
c & Acopiar materials per al muntatge i manteniment de sistemes
informàtics & 5 \% \\
d & Realitzar operacions auxiliars de muntatge de sistemes
microinformàtics & 10 \% \\
e & Realitzar operacions de manteniment bàsic de sistemes
microinformàtics & 10 \% \\
f & Realitzar operacions d'emmagatzematge i transport de perifèrics i
sistemes & 5 \% \\
g & Verificar sistemes i instal·lacions informàtiques segons
procediments establerts & 5 \% \\
h & Muntar canalitzacions i cablatge de dades amb criteris de qualitat i
seguretat & 5 \% \\
i & Realitzar el cablejat de xarxes locals segons tècniques
normalitzades & 5 \% \\
j & Manejar eines de l/'entorn d'usuari i dispositius d'emmagatzematge
d'informació & 5 \% \\
k & Resoldre problemes previsibles amb suport científic i tecnològic & 3
\% \\
l & Fomentar hàbits saludables en l'entorn personal i social & 2 \% \\
m & Aplicar accions de conservació del medi ambient & 2 \% \\
n & Utilitzar tecnologies per a l'autonomia i aprenentatge permanent & 3
\% \\
o & Comunicar-se amb claredat en diversos entorns i mitjans & 3 \% \\
p & Comunicar-se en llengua estrangera en situacions habituals & 2 \% \\
q & Explicar fets o fenòmens socials amb precisió lingüística & 2 \% \\
r & Adaptar-se a canvis tecnològics i organitzatius en l'activitat
professional & 3 \% \\
s & Complir tasques pròpies amb responsabilitat i eficiència & 4 \% \\
t & Comunicar-se i treballar en equip respectant l'autonomia i
competència dels altres & 3 \% \\
u & Aplicar mesures de prevenció de riscos laborals i seguretat & 3
\% \\
v & Complir normes de qualitat, accessibilitat i disseny universal & 2
\% \\
w & Actuar amb iniciativa i responsabilitat en la tria de procediments
professionals & 2 \% \\
x & Exercir els drets i obligacions professionals amb participació en la
vida social i econòmica & 1 \% \\
\textbf{Total} & & \textbf{100 \%} \\
\bottomrule
\end{longtable}

Aquesta ponderació serà tinguda en compte en la programació d'aula, en
les situacions d'aprenentatge i en el disseny de projectes
intermodulars, amb l'objectiu de formar titulats i titulades preparats
per a \textbf{l'ocupabilitat real i el desenvolupament professional
sostenible} dins del sector TIC del territori.

\hypertarget{contribuciuxf3-de-cada-muxf2dul-a-les-competuxe8ncies-professionals-del-cicle}{%
\section{4. Contribució de cada mòdul a les competències professionals
del
cicle}\label{contribuciuxf3-de-cada-muxf2dul-a-les-competuxe8ncies-professionals-del-cicle}}

Per a la coordinació del treball educatiu, és necessari \textbf{estudiar
com cada mòdul contribueix al desenvolupament de les competències
professionals} del títol. Aquesta visió integrada permet una
planificació coherent i compartida entre els docents implicats, i ajuda
a garantir que totes les competències requerides siguen abordades de
manera progressiva i equilibrada al llarg del cicle formatiu.

Aquest enfocament també permet, en un nivell més avançat de programació,
\textbf{identificar resultats d'aprenentatge (RA) clau} que, pel seu
caràcter transversal o pel seu impacte professional, requerisquen una
atenció especial dins de cada mòdul. Així, el procés de programació pot
ajustar-se per garantir que cap competència quede desatesa i que es
desenvolupen de manera efectiva aquelles que són més rellevants per a la
inserció laboral i el desenvolupament personal de l'alumnat.

Per tal d'arribar a aquest nivell de concreció, es proposa estructurar
una \textbf{taula de contribució competencial}, en què es relacione cada
mòdul amb les competències professionals del perfil. Aquesta associació
s'establix a partir de les orientacions del Reial decret que regula el
títol (RD 356/2014), i pot ser revisada o ajustada per l'equip educatiu
si ho considera oportú, d'acord amb les característiques del context i
del grup classe.

La \textbf{competència general} del títol és la següent:

\begin{quote}
\emph{/``Realitzar operacions auxiliars de muntatge i manteniment de
sistemes microinformàtics, perifèrics i xarxes de comunicació de dades,
i de tractament, reproducció i arxiu de documents, operant amb la
qualitat indicada i actuant en condicions de seguretat i de protecció
ambiental amb responsabilitat i iniciativa personal, i comunicant-se de
forma oral i escrita en llengua castellana, i si escau, en la llengua
cooficial pròpia, així com en alguna llengua estrangera./''}
\end{quote}

\hypertarget{taula-de-corresponduxe8ncia-entre-muxf2duls-i-competuxe8ncies-professionals}{%
\subsection{4.1. Taula de correspondència entre mòduls i competències
professionals}\label{taula-de-corresponduxe8ncia-entre-muxf2duls-i-competuxe8ncies-professionals}}

A continuació es presenta la relació entre els \textbf{mòduls
professionals} del cicle i les \textbf{competències professionals,
personals i socials} que contribueixen a desenvolupar:

\begin{longtable}[]{@{}ll@{}}
\toprule
\textbf{Mòdul Professional} & \textbf{Competències Professionals,
Personals i Socials Desenvolupades} \\
\midrule
\endhead
3029/. Muntatge i manteniment de sistemes i components informàtics & c),
d), e), f), g), h), i), j) \\
3030/. Operacions auxiliars per a la configuració i explotació & a),
j) \\
3016/. Instal·lació i manteniment de xarxes per a transmissió de dades &
h), i) \\
3031/. Ofimàtica i arxiu de documents & a), b), j) \\
3011/. Comunicació i societat I & m), n), ñ), o), p) \\
3012/. Comunicació i societat II & q), r), s), t), u), v), w) \\
3009/. Ciències aplicades I & j), k), l), m) \\
3010/. Ciències aplicades II & q), r), s), t), u), v), w), x) \\
3033/. Formació en centres de treball (FCT) & Totes les competències
transversals i tècniques en context real \\
\bottomrule
\end{longtable}

\begin{quote}
Aquesta taula és un instrument fonamental per a la \textbf{coordinació
entre docents}, la programació compartida de situacions d'aprenentatge i
la detecció de possibles buits competencials en el desenvolupament del
currículum.
\end{quote}

\hypertarget{utilitzaciuxf3-diduxe0ctica}{%
\subsection{4.2. Utilització
didàctica}\label{utilitzaciuxf3-diduxe0ctica}}

El claustre i l'equip docent poden utilitzar aquesta informació per:

\begin{itemize}
\tightlist
\item
  Dissenyar activitats i projectes integrats amb una visió competencial
\item
  Assignar responsabilitats específiques de seguiment i avaluació per
  competència
\item
  Coordinar l'acció tutorial i les activitats d'orientació professional
\item
  Planificar de manera col·laborativa els \textbf{projectes
  intermodulars} i les \textbf{situacions d'aprenentatge}
\end{itemize}

Amb tot això, el \textbf{PCCF esdevé una eina viva}, en constant
revisió, que assegura la cohesió del treball educatiu i l'alineació de
les actuacions docents amb els objectius del sistema de Formació
Professional i les necessitats de l'entorn.

\hypertarget{estructura-modular-i-transiciuxf3-del-curruxedculum}{%
\subsubsection{4.3. Estructura modular i transició del
currículum}\label{estructura-modular-i-transiciuxf3-del-curruxedculum}}

Actualment, el cicle \textbf{Tècnic Bàsic en Informàtica d'Oficina}
presenta la següent estructura modular i horària, vigent \textbf{fins al
curs 2024-2025}. A partir del curs 2025-2026, s'implantarà una nova
estructura que reflecteix els canvis introduïts pel nou marc normatiu.

\hypertarget{estructura-vigent-fins-al-curs-2024-2025}{%
\paragraph{\texorpdfstring{\textbf{Estructura vigent fins al curs
2024-2025}}{Estructura vigent fins al curs 2024-2025}}\label{estructura-vigent-fins-al-curs-2024-2025}}

\begin{longtable}[]{@{}llll@{}}
\toprule
\textbf{Curs} & \textbf{Mòdul Professional} & \textbf{Hores setmanals} &
\textbf{Hores totals} \\
\midrule
\endhead
\textbf{1r} & 3031/. Ofimàtica i arxiu de documents & 9 & 300 \\
& 3029/. Muntatge i manteniment de sistemes i components informàtics & 9
& 290 \\
& 3009/. Ciències aplicades I & 5 & 158 \\
& 3011/. Comunicació i societat I & 5 & 158 \\
& Tutoria & 1 & 34 \\
& CV0005. Formació i Orientació Laboral I & 1 & 30 \\
& \textbf{Total 1r curs} & \textbf{30} & \textbf{970} \\
\textbf{2n} & 3016/. Instal·lació i manteniment de xarxes per a
transmissió de dades & 10 & 255 \\
& 3030/. Operacions auxiliars per a la configuració i l'explotació & 6 &
155 \\
& 3019/. Ciències aplicades II & 6 & 158 \\
& 3012/. Comunicació i societat II & 6 & 158 \\
& Tutoria & 1 & 34 \\
& CV0006. Formació i Orientació Laboral II & 1 & 30 \\
& 3033/. Formació en centres de treball & /- & 240 \\
& \textbf{Total 2n curs} & \textbf{30} & \textbf{1.030} \\
& \textbf{Total cicle} & \textbf{60} & \textbf{2.000} \\
\bottomrule
\end{longtable}

\begin{center}\rule{0.5\linewidth}{0.5pt}\end{center}

\hypertarget{nova-estructura-a-partir-del-curs-2025-2026}{%
\paragraph{\texorpdfstring{\textbf{Nova estructura a partir del curs
2025-2026}}{Nova estructura a partir del curs 2025-2026}}\label{nova-estructura-a-partir-del-curs-2025-2026}}

\begin{longtable}[]{@{}lllll@{}}
\toprule
\textbf{Curs} & \textbf{Codi} & \textbf{Mòdul Professional} &
\textbf{Hores setmanals} & \textbf{Hores totals} \\
\midrule
\endhead
\textbf{1r} & 3029 & Muntatge i manteniment de sistemes i components
informàtics & 9 & 299 \\
& 3031 & Ofimàtica i arxiu de documents & 9 & 299 \\
& 3161 & Comunicació i Ciències Socials I & 4 & 133 \\
& 3163 & Ciències aplicades I & 4 & 133 \\
& TU01CF & Tutoria Primer & 2 & 69 \\
& 3159p & Itinerari personal per a l'ocupabilitat 1r & 2 & 67 \\
& & \textbf{Total 1r curs} & \textbf{30} & \textbf{1.000} \\
\textbf{2n} & 3016 & Instal·lació i manteniment de xarxes de comunicació
& 10 & 332 \\
& 3030 & Configuració i explotació de sistemes informàtics & 6 & 200 \\
& 3159 & Itinerari personal per a l'ocupabilitat & 1 & 34 \\
& 3162 & Comunicació i Ciències Socials II & 5 & 166 \\
& 3164 & Ciències aplicades II & 5 & 166 \\
& TU02CF & Tutoria Segon & 1 & 35 \\
& 3160972 & Projecte intermodular d'aprenentatge col·laboratiu & 2 &
67 \\
& & \textbf{Total 2n curs} & \textbf{30} & \textbf{1.000} \\
& & \textbf{Total cicle} & \textbf{60} & \textbf{2.000} \\
\bottomrule
\end{longtable}

\begin{center}\rule{0.5\linewidth}{0.5pt}\end{center}

Aquesta nova estructura reforça el \textbf{component competencial,
transversal i integrat del currículum}, amb una clara aposta per la
coordinació entre mòduls, la personalització de l'itinerari formatiu i
el treball per projectes. L'equip docent haurà d'adaptar les seues
programacions i estratègies metodològiques per tal de garantir
l'aplicació efectiva d'aquest nou model.

\hypertarget{comparativa-entre-lestructura-vigent-i-la-nova-proposta-curricular}{%
\subsection{4.4. Comparativa entre l'estructura vigent i la nova
proposta
curricular}\label{comparativa-entre-lestructura-vigent-i-la-nova-proposta-curricular}}

La següent taula resumeix de manera comparativa les diferències entre
l'estructura \textbf{vigent fins al curs 2024-2025} i la \textbf{nova
proposta curricular a partir del curs 2025-2026} per al cicle de
\textbf{Tècnic Bàsic en Informàtica d'Oficina}:

\begin{longtable}[]{@{}lll@{}}
\toprule
\textbf{Aspecte} & \textbf{Estructura vigent (fins 2024-2025)} &
\textbf{Nova estructura (a partir de 2025-2026)} \\
\midrule
\endhead
Nombre total de mòduls & 12 mòduls & 12 mòduls (amb codis i
denominacions revisades) \\
Projecte intermodular & Apareix només a 2n curs com a mòdul específic &
Es manté i es reforça com a element vertebrador \\
Itinerari personal per a l'ocupabilitat & Present als dos cursos & Es
manté, però amb integració més activa en el PCCF \\
Tutoria & Dos mòduls (1r i 2n) & Es manté igual \\
Comunicació i Ciències Socials & Dues fases (I i II) & Es manté, amb
ajustos metodològics \\
Ciències aplicades & Dues fases (I i II) & Es manté, amb orientació cap
a contextos professionals \\
Denominació de mòduls tècnics & Formulació genèrica & Denominació més
precisa i funcional: configuració, xarxes/\ldots{} \\
Distribució horària & 30 h setmanals / 2.000 h totals & Sense canvis
(mateix volum global) \\
Orientació competencial & Implícita i a través d'objectius & Explícita,
integrada i centrada en resultats d'aprenentatge (RA) \\
Enfocament metodològic & Tradicional amb algunes iniciatives actives &
Actiu, per projectes, personalitzat i transversal \\
\bottomrule
\end{longtable}

\hypertarget{consideracions-per-a-la-implementaciuxf3}{%
\subsection{4.5. Consideracions per a la
implementació}\label{consideracions-per-a-la-implementaciuxf3}}

La nova proposta curricular comporta un \textbf{canvi de paradigma} en
la manera com es planifica, s'imparteix i s'avalua el currículum de la
Formació Professional Bàsica. Les principals implicacions per al
desenvolupament del \textbf{PCCF} són:

\begin{itemize}
\tightlist
\item
  Necessitat d'una \textbf{coordinació didàctica més estreta} entre
  docents dels diferents mòduls.
\item
  Incorporació de \textbf{metodologies actives} i de situacions
  d'aprenentatge contextualitzades.
\item
  Major \textbf{centralitat del projecte intermodular}, que esdevé eix
  transversal del procés formatiu.
\item
  Integració efectiva de les \textbf{competències per a l'ocupabilitat},
  amb accions concretes des de cada mòdul.
\item
  Adaptació progressiva de les programacions a un model \textbf{orientat
  a l'assoliment de RA} i no només de continguts.
\end{itemize}

Aquest context reforça la funció estratègica del \textbf{PCCF com a
instrument de planificació col·lectiva}, alineat amb els objectius del
centre, el PEC i les exigències de la nova legislació educativa.

\hypertarget{contribuciuxf3-de-cada-muxf2dul-a-les-competuxe8ncies-per-a-locupabilitat}{%
\section{5. Contribució de cada mòdul a les competències per a
l/'ocupabilitat}\label{contribuciuxf3-de-cada-muxf2dul-a-les-competuxe8ncies-per-a-locupabilitat}}

Igual que en l'apartat anterior, és important identificar \textbf{quines
competències per a l'ocupabilitat es treballaran en cada mòdul
professional}, per tal d'integrar-les de manera coherent i efectiva a
l/'hora de programar.

Aquestes competències, que inclouen aspectes com la responsabilitat,
l'autonomia, el treball en equip o l'adaptabilitat, \textbf{no tenen una
relació directa amb les habilitats tècniques pròpies de la professió}, i
per això no resulta convenient aplicar una ponderació percentual com es
fa amb les competències professionals.

Al contrari, es recomana que \textbf{l'equip docent, en funció de la
seua experiència i del disseny de les activitats d'aprenentatge,
decidisca en quins mòduls és més adient treballar cada competència per a
l'ocupabilitat}. Aquesta decisió s'hauria de prendre col·lectivament,
tenint en compte el context, l'alumnat i els escenaris d'aplicació
professional simulats o reals.

Aquestes competències són fonamentals per a l'\textbf{acollida, la
permanència i el progrés en l'àmbit laboral}, i han d'estar
\textbf{reflectides de manera clara i visible en les rúbriques, llistes
de control, activitats d'autoavaluació i qualsevol altre instrument
emprat per a l'avaluació} dels mòduls que les treballen. Aquesta
inclusió permet garantir que el procés d'avaluació valore \textbf{tant
el desenvolupament tècnic com les actituds i capacitats personals i
socials} que configuren un perfil professional complet.

\hypertarget{relaciuxf3-orientativa-entre-muxf2duls-i-competuxe8ncies-per-a-locupabilitat}{%
\subsection{5.1. Relació orientativa entre mòduls i competències per a
l'ocupabilitat}\label{relaciuxf3-orientativa-entre-muxf2duls-i-competuxe8ncies-per-a-locupabilitat}}

La següent taula ofereix una proposta inicial de \textbf{distribució de
les competències per a l'ocupabilitat} entre els diferents mòduls del
cicle:

\begin{longtable}[]{@{}ll@{}}
\toprule
\textbf{Mòdul Professional} & \textbf{Competències per a l'ocupabilitat
desenvolupades} \\
\midrule
\endhead
3029/. Muntatge i manteniment de sistemes i components informàtics & t),
u), v), w), x), y), z) \\
3031/. Ofimàtica i arxiu de documents & t), u), v), w), x), y), z) \\
3016/. Instal·lació i manteniment de xarxes per a transmissió de dades &
t), u), v), w), x), y), z) \\
3030/. Operacions auxiliars per a la configuració i l'explotació & t),
u), v), w), x), y), z) \\
3161 / 3162. Comunicació i Ciències Socials I i II & s), t), u), v), w),
z) \\
3163 / 3164. Ciències aplicades I i II & t), u), v), w), x), y), z) \\
3159/. Itinerari personal per a l'ocupabilitat & s), t), u), v), w), x),
y), z) \\
3160972/. Projecte intermodular d'aprenentatge col·laboratiu & t), u),
v), w), y), z) \\
Tutoria (TU01CF / TU02CF) & Seguiment actitudinal i suport al
desenvolupament transversal \\
\bottomrule
\end{longtable}

\begin{quote}
Aquesta distribució hauria de ser revisada per l'equip docent en cada
curs acadèmic, d'acord amb les programacions, els perfils d'alumnat i
els objectius del projecte intermodular. Les competències per a
l'ocupabilitat s'han d'integrar de forma natural en la dinàmica d'aula i
han de ser \textbf{avaluables mitjançant indicadors clars i
observables}.
\end{quote}

Perquè això siga possible, cal:

\begin{itemize}
\tightlist
\item
  Planificar situacions d'aprenentatge on aquestes competències
  apareguen de manera contextualitzada i significativa.
\item
  Incorporar-les explícitament en els \textbf{instruments d'avaluació}
  del mòdul (rúbriques, escales de valoració, llistes de control,
  autoavaluacions, etc.).
\item
  Fer ús de \textbf{metodologies actives i cooperatives} que faciliten
  el seu desplegament (projectes, tasques compartides, simulacions,
  etc.).
\item
  Establir espais de coordinació entre docents per \textbf{acordar
  evidències comunes}, seqüenciar-ne el treball i \textbf{garantir la
  coherència avaluadora}.
\end{itemize}

\hypertarget{ruxfabrica-dexemple-competuxe8ncia-treball-en-equip-competuxe8ncia-t}{%
\subsection{5.2. Rúbrica d'exemple: Competència ``Treball en equip''
(competència
t)}\label{ruxfabrica-dexemple-competuxe8ncia-treball-en-equip-competuxe8ncia-t}}

La següent rúbrica mostra un exemple pràctic per a avaluar la
competència \textbf{t) Comunicar-se i treballar en equip respectant
l'autonomia i competència dels altres} dins d'una situació
d'aprenentatge col·laborativa:

\begin{longtable}[]{@{}lllll@{}}
\toprule
\textbf{Indicador} & \textbf{Nivell 1} (Iniciació) & \textbf{Nivell 2}
(En procés) & \textbf{Nivell 3} (Assolit) & \textbf{Nivell 4}
(Excel·lent) \\
\midrule
\endhead
Participa activament en les tasques col·lectives & Només quan se li
demana & Participa de manera irregular & Participa de manera regular &
Participa amb iniciativa i constància \\
Respecta les opinions i funcions del grup & Té dificultats per
acceptar-les & Les accepta amb ajuda & Les respecta i coopera activament
& Ajuda a integrar les aportacions dels altres \\
Es comunica de forma clara i efectiva amb el grup & Amb dificultats i
poc claredat & Amb certa claredat i correcció & Amb claredat i adequació
& Amb fluïdesa, assertivitat i empatia \\
Assumeix responsabilitats i compleix els acords de grup & Evita
responsabilitats & Assumeix algunes tasques & Assumeix tasques de manera
responsable & Assumeix lideratges i facilita la cohesió del grup \\
\bottomrule
\end{longtable}

\begin{quote}
Aquesta rúbrica pot adaptar-se a altres mòduls i a altres competències
(com la responsabilitat, la iniciativa o l'autonomia), mantenint sempre
la claredat dels indicadors i la relació amb les situacions
d'aprenentatge concretes.
\end{quote}

La inclusió d'aquests instruments en les programacions d'aula, així com
el seu ús sistemàtic per part del professorat, permetrà consolidar una
\textbf{avaluació integral i equitativa}, que tinga en compte
\textbf{tant el desenvolupament tècnic com el creixement personal i
social} de l'alumnat.

A més, cal tindre en compte que \textbf{l'equip docent es reuneix de
manera setmanal}, fet que facilita el \textbf{seguiment compartit del
desenvolupament de les competències per a l'ocupabilitat}. Aquestes
reunions permeten:

\begin{itemize}
\tightlist
\item
  Analitzar l'evolució de l'alumnat en relació amb aquestes competències
  transversals.
\item
  Compartir evidències d'aprenentatge observades en els diferents
  mòduls.
\item
  Ajustar o redefinir indicadors i estratègies d'intervenció educativa
  segons les necessitats reals del grup.
\item
  Coordinar accions específiques per al desenvolupament d'algunes
  competències en situacions comunes (com el projecte intermodular o
  activitats col·laboratives).
\end{itemize}

Aquest \textbf{seguiment sistemàtic, reflexiu i compartit} garanteix una
visió integral de l'alumnat, i assegura que la \textbf{avaluació de les
competències per a l'ocupabilitat} siga \textbf{coherent, contínua i
orientada a la millora}.

A més, les conclusions de les reunions setmanals es recolliran de manera
estructurada en actes i informes de seguiment pedagògic, que permetran
fer un \textbf{retorn regular a l'alumnat}, així com ajustar, si cal,
les intervencions educatives i les estratègies metodològiques.

\hypertarget{enfocaments-diduxe0ctics-i-metodoluxf2gics}{%
\section{6. Enfocaments didàctics i
metodològics}\label{enfocaments-diduxe0ctics-i-metodoluxf2gics}}

El cicle de Formació Professional Bàsica en Informàtica d'Oficina a
l'IES Jaume II el Just es fonamenta en un \textbf{enfocament actiu,
contextualitzat i professionalitzador}, alineat amb els principis
recollits a l'article 13 del \textbf{Reial decret 659/2023}, que promou
la incorporació de \textbf{metodologies actives} en la docència dels
cicles formatius.

Aquest enfocament no es limita a una declaració d'intencions, sinó que
es \textbf{materialitza en un conjunt de principis metodològics
consensuats} per l'equip docent, amb caràcter vinculant per a totes les
programacions del cicle. El consens metodològic ha estat treballat en
les reunions setmanals del claustre del cicle i suposa un compromís
explícit amb un model pedagògic transformador.

\hypertarget{principis-metodoluxf2gics-acordats}{%
\subsection{6.1. Principis metodològics
acordats}\label{principis-metodoluxf2gics-acordats}}

L'equip docent acorda aplicar de forma transversal les següents
estratègies metodològiques:

\begin{itemize}
\tightlist
\item
  \textbf{Aprenentatge basat en projectes (ABP)}: l'alumnat desenvolupa
  \textbf{projectes interdisciplinars} reals o simulats, com ara:

  \begin{itemize}
  \tightlist
  \item
    \emph{Projecte de reciclatge de plàstics i impressió 3D} (Ciències
    Aplicades II, IMX i Comunicació i Societat II)
  \item
    \emph{Projecte ``Aprenem de la Pluja''}, vinculat a la DANA de 2024,
    orientat a la restauració d'equipament TIC i dades.
  \item
    \emph{Participació en la Fira Experimenta}, amb projectes de
    caràcter científic-tecnològic de divulgació.
  \end{itemize}
\item
  \textbf{Aprenentatge-Servei (ApS)}: com a eix transversal, alguns
  projectes integren una dimensió social i de servei comunitari, com el
  \textbf{manteniment i instal·lació de xarxes} al propi centre per part
  de l'alumnat, amb finalitats reals i aplicació immediata.
\item
  \textbf{Aprenentatge col·laboratiu}: foment del treball en equip
  mitjançant activitats grupals estructurades, ús d'eines de gestió
  col·laborativa i \textbf{responsabilitat compartida} en tasques i
  rols.
\item
  \textbf{Contextualització professional i realisme pedagògic}: les
  situacions d'aprenentatge s'inspiren en l'àmbit laboral real i
  aprofiten el context del centre i l'entorn socioeconòmic.
\item
  \textbf{Personalització de l'aprenentatge i enfocament inclusiu},
  adaptant les activitats als ritmes, necessitats i potencialitats de
  l'alumnat.
\item
  \textbf{Ús pedagògic de les TIC}: integració d'eines digitals tant per
  a la producció com per a la comunicació i l'organització del treball.
\item
  \textbf{Avaluació contínua i formativa}, basada en evidències
  d'aprenentatge, rúbriques competencials, observació directa,
  autoavaluacions i portafolis.
\end{itemize}

\hypertarget{coheruxe8ncia-seguiment-i-compromuxeds-docent}{%
\subsection{6.2. Coherència, seguiment i compromís
docent}\label{coheruxe8ncia-seguiment-i-compromuxeds-docent}}

Aquest marc metodològic consensuat és de \textbf{compliment obligat per
a tot el professorat del cicle}, i s'aplicarà tant a nivell de mòdul com
en el desenvolupament del \textbf{Projecte Intermodular d'Aprenentatge
Col·laboratiu}.

Les \textbf{reunions setmanals de l'equip docent} tindran com a funció,
entre altres, fer \textbf{seguiment del desplegament metodològic},
ajustar estratègies si cal i garantir que es manté la coherència entre
els mòduls i els objectius del PCCF.

Aquest enfocament metodològic pretén oferir una resposta eficaç als
desafiaments del món laboral i educatiu actual, \textbf{convertint
l'alumnat en protagonista actiu del seu aprenentatge} i facilitant la
seua integració futura en entorns professionals reals.

\hypertarget{organitzaciuxf3-i-distribuciuxf3-dels-muxf2duls-professionals}{%
\section{7. Organització i distribució dels mòduls
professionals}\label{organitzaciuxf3-i-distribuciuxf3-dels-muxf2duls-professionals}}

D'acord amb l'article 11 del \textbf{Reial decret 659/2023}, els centres
de Formació Professional poden organitzar els mòduls de manera flexible,
sempre que es respecten els resultats d'aprenentatge de cada un i el
currículum oficial.

Amb aquesta finalitat, l'equip docent del cicle de \textbf{Formació
Professional Bàsica en Informàtica d'Oficina} ha acordat una
organització interna que \textbf{afavoreix el treball interdisciplinari
dins de cada curs}, mitjançant agrupacions pedagògiques que permeten
desenvolupar \textbf{projectes comuns, integració competencial i
metodologies actives} com l'ABP i l'ApS.

\hypertarget{organitzaciuxf3-dels-muxf2duls-en-1r-curs}{%
\subsection{\texorpdfstring{7.1. Organització dels mòduls en \textbf{1r
curs}}{7.1. Organització dels mòduls en 1r curs}}\label{organitzaciuxf3-dels-muxf2duls-en-1r-curs}}

\begin{longtable}[]{@{}
  >{\raggedright\arraybackslash}p{(\columnwidth - 4\tabcolsep) * \real{0.2968}}
  >{\raggedright\arraybackslash}p{(\columnwidth - 4\tabcolsep) * \real{0.3097}}
  >{\raggedright\arraybackslash}p{(\columnwidth - 4\tabcolsep) * \real{0.3806}}@{}}
\toprule
\begin{minipage}[b]{\linewidth}\raggedright
\textbf{Àmbit de treball integrador}
\end{minipage} & \begin{minipage}[b]{\linewidth}\raggedright
\textbf{Mòduls implicats}
\end{minipage} & \begin{minipage}[b]{\linewidth}\raggedright
\textbf{Projectes o actuacions vinculades}
\end{minipage} \\
\midrule
\endhead
\textbf{Tecnologia i manteniment bàsic} & /- Muntatge i manteniment de
sistemes/ - Ciències Aplicades I & Projecte: Simulació d'un taller TIC
escolar \\
\textbf{Ofimàtica i documentació digital} & /- Ofimàtica i arxiu de
documents/ - Comunicació i Ciències Socials I & Projecte: Creació de
materials digitals i fitxes manuals \\
\textbf{Ocupabilitat i desenvolupament personal} & /- Itinerari personal
per a l'ocupabilitat I/ - Tutoria I & Dossier de perfil professional
inicial \\
\bottomrule
\end{longtable}

\hypertarget{organitzaciuxf3-dels-muxf2duls-en-2n-curs}{%
\subsection{\texorpdfstring{7.2. Organització dels mòduls en \textbf{2n
curs}}{7.2. Organització dels mòduls en 2n curs}}\label{organitzaciuxf3-dels-muxf2duls-en-2n-curs}}

\begin{longtable}[]{@{}
  >{\raggedright\arraybackslash}p{(\columnwidth - 4\tabcolsep) * \real{0.2781}}
  >{\raggedright\arraybackslash}p{(\columnwidth - 4\tabcolsep) * \real{0.2899}}
  >{\raggedright\arraybackslash}p{(\columnwidth - 4\tabcolsep) * \real{0.4201}}@{}}
\toprule
\begin{minipage}[b]{\linewidth}\raggedright
\textbf{Àmbit de treball integrador}
\end{minipage} & \begin{minipage}[b]{\linewidth}\raggedright
\textbf{Mòduls implicats}
\end{minipage} & \begin{minipage}[b]{\linewidth}\raggedright
\textbf{Projectes o actuacions vinculades}
\end{minipage} \\
\midrule
\endhead
\textbf{Xarxes i comunicacions TIC} & /- Instal·lació i manteniment de
xarxes/ - Operacions auxiliars per a la configuració & Projecte APS:
Manteniment real de la xarxa del centre \\
\textbf{Tecnologia i sostenibilitat} & /- Ciències Aplicades II/ -
Projecte intermodular & Projecte: Reciclatge de plàstic i impressió
3D \\
\textbf{Comunicació i gestió documental avançada} & /- Comunicació i
Ciències Socials II/ - Ofimàtica aplicada (recuperació i anàlisi) &
Projecte: Memòries tècniques i comunicació digital \\
\textbf{Ocupabilitat i orientació professional} & /- Itinerari personal
per a l'ocupabilitat II/ - Tutoria II & Projecte: Dossier professional +
CV + simulació d'entrevista laboral \\
\bottomrule
\end{longtable}

\begin{quote}
El mòdul \textbf{Projecte intermodular d'aprenentatge col·laboratiu}
s'integra de manera transversal als projectes del segon curs, actuant
com a eix vertebrador de la metodologia ABP i facilitant la coordinació
docent.
\end{quote}

\hypertarget{objectius-daquesta-reorganitzaciuxf3}{%
\subsection{7.3. Objectius d'aquesta
reorganització}\label{objectius-daquesta-reorganitzaciuxf3}}

Aquesta estructura respon a la voluntat de:

\begin{itemize}
\tightlist
\item
  Potenciar la \textbf{coherència metodològica i competencial} dins de
  cada curs.
\item
  Desenvolupar \textbf{projectes amb sentit real i integrats} dins de
  l'horari lectiu.
\item
  Facilitar la \textbf{coordinació entre docents} de cada nivell per
  planificar i avaluar conjuntament.
\item
  Fomentar una \textbf{transició progressiva} cap a l'entorn
  professional i la inserció laboral.
\item
  Optimitzar l'ús del \textbf{Projecte Intermodular} com a espai de
  síntesi i treball col·laboratiu.
\end{itemize}

Aquesta reorganització \textbf{no altera la càrrega horària oficial} del
cicle, sinó que reconfigura \textbf{lògiques de treball didàctic} per a
afavorir l/'aprenentatge significatiu, transversal i amb sentit per a
l'alumnat.

\hypertarget{organitzaciuxf3-especuxedfica-del-muxf2dul-comunicaciuxf3-i-societat}{%
\subsubsection{7.4. Organització específica del mòdul ``Comunicació i
Societat''}\label{organitzaciuxf3-especuxedfica-del-muxf2dul-comunicaciuxf3-i-societat}}

El mòdul de \textbf{Comunicació i Societat} presenta una estructura
\textbf{singular} dins la Formació Professional Bàsica, ja que
\textbf{integra de manera transversal continguts de diverses matèries de
caràcter general}:

\begin{itemize}
\tightlist
\item
  \textbf{Llengua Castellana}
\item
  \textbf{Llengua Valenciana}
\item
  \textbf{Ciències Socials}
\item
  \textbf{Llengua Estrangera (Anglés)}
\end{itemize}

A diferència d'altres mòduls de caràcter professional, aquest mòdul
exigeix una \textbf{planificació didàctica coordinada entre diversos
docents}, els quals han de garantir conjuntament el desenvolupament de
tots els resultats d'aprenentatge vinculats als àmbits implicats.

Amb aquesta finalitat, el centre ha establit una \textbf{organització
interna específica} del mòdul, basada en els següents criteris
pedagògics:

\begin{itemize}
\tightlist
\item
  \textbf{Coordinació metodològica transversal} mitjançant programacions
  compartides i reunions de seguiment entre els docents dels àmbits
  lingüístic, social i estranger.
\item
  \textbf{Distribució equilibrada dels continguts entre primer i segon
  curs}, assegurant una progressió didàctica coherent.
\item
  \textbf{Integració funcional de l'anglés} en tasques
  contextualitzades: simulació d'entrevistes, elaboració de documents,
  presentacions orals, etc.
\item
  \textbf{Orientació competencial i transversal}, vinculada a les
  competències comunicatives, socials, cíviques i per a l'ocupabilitat.
\item
  \textbf{Participació activa en projectes intermodulars}, on la
  dimensió comunicativa té un paper fonamental: redacció de memòries,
  sensibilització, exposicions orals, etc.
\end{itemize}

Aquest enfocament garanteix el desenvolupament d'habilitats de
comunicació bàsica, de reflexió crítica sobre la realitat social i de
participació responsable a la comunitat educativa.

Els docents implicats assumeixen el compromís de treballar amb
\textbf{criteris d'avaluació consensuats}, i d'aplicar
\textbf{instruments compartits d'avaluació formativa i competencial}, en
coordinació amb el conjunt del professorat del cicle i dins del marc del
\textbf{Projecte d'Acció Curricular (PAC)}.

\begin{center}\rule{0.5\linewidth}{0.5pt}\end{center}

\hypertarget{organitzaciuxf3-del-muxf2dul-comunicaciuxf3-i-societat-i-codi-3011}{%
\subsubsection{7.4.1 Organització del mòdul Comunicació i Societat I
(Codi
3011)}\label{organitzaciuxf3-del-muxf2dul-comunicaciuxf3-i-societat-i-codi-3011}}

Aquest mòdul, situat en \textbf{primer curs}, integra continguts
inicials de llengua castellana, llengua anglesa i ciències socials. Els
seus resultats d'aprenentatge (RA) tenen com a eixos:

\begin{itemize}
\tightlist
\item
  L'anàlisi de les societats prehistòriques i de l'Edat Antiga, i les
  seues relacions amb l'entorn natural i artístic.
\item
  La construcció de l'espai europeu fins a les primeres transformacions
  industrials.
\item
  L'adquisició de competències bàsiques d'expressió i comprensió oral i
  escrita en castellà.
\item
  La iniciació a la literatura en llengua castellana.
\item
  La comunicació oral i escrita funcional en anglès en situacions
  quotidianes i escolars.
\end{itemize}

\begin{center}\rule{0.5\linewidth}{0.5pt}\end{center}

\hypertarget{organitzaciuxf3-del-muxf2dul-comunicaciuxf3-i-societat-ii-codi-3012}{%
\subsubsection{7.4.2 Organització del mòdul Comunicació i Societat II
(Codi
3012)}\label{organitzaciuxf3-del-muxf2dul-comunicaciuxf3-i-societat-ii-codi-3012}}

En \textbf{segon curs}, el mòdul aprofundeix en els mateixos àmbits però
des d'un enfocament més complex i contextualitzat. Els RA s'orienten a:

\begin{itemize}
\tightlist
\item
  Comprendre l'evolució de les societats contemporànies i el sistema
  democràtic.
\item
  Comunicar-se amb claredat i correcció en llengua castellana en
  contextos socials i laborals.
\item
  Interpretar i produir textos literaris des del segle XIX fins a
  l'actualitat.
\item
  Utilitzar l'anglés en interaccions senzilles de caràcter professional
  i personal, amb vocabulari específic i estructures bàsiques.
\end{itemize}

\begin{center}\rule{0.5\linewidth}{0.5pt}\end{center}

\hypertarget{comparativa-de-resultats-daprenentatge-comunicaciuxf3-i-societat-i-vs-ii}{%
\subsubsection{7.4.3 Comparativa de Resultats d'Aprenentatge:
Comunicació i Societat I vs
II}\label{comparativa-de-resultats-daprenentatge-comunicaciuxf3-i-societat-i-vs-ii}}

\begin{longtable}[]{@{}
  >{\raggedright\arraybackslash}p{(\columnwidth - 4\tabcolsep) * \real{0.2047}}
  >{\raggedright\arraybackslash}p{(\columnwidth - 4\tabcolsep) * \real{0.3275}}
  >{\raggedright\arraybackslash}p{(\columnwidth - 4\tabcolsep) * \real{0.4561}}@{}}
\toprule
\begin{minipage}[b]{\linewidth}\raggedright
\textbf{Àmbit}
\end{minipage} & \begin{minipage}[b]{\linewidth}\raggedright
\textbf{Comunicació i Societat I (3011)}
\end{minipage} & \begin{minipage}[b]{\linewidth}\raggedright
\textbf{Comunicació i Societat II (3012)}
\end{minipage} \\
\midrule
\endhead
\textbf{Ciències Socials} & RA1: Societats prehistòriques i antigues/
RA2: Edat Mitjana, moderna i construcció europea & RA1: Societats
contemporànies i evolució històrica/ RA2: Democràcia, Drets Humans i
participació ciutadana \\
\textbf{Llengua Castellana (oral)} & RA3: Comunicació oral bàsica i
escolta activa & RA3: Comunicació oral estructurada i expressió adequada
a contextos formals \\
\textbf{Llengua Castellana (escrita)} & RA4: Producció de textos breus i
lectura comprensiva/ RA5: Iniciació a la literatura clàssica & RA4:
Producció escrita de textos més complexos/ RA5: Literatura contemporània
i anàlisi crítica \\
\textbf{Llengua Anglesa} & RA6: Expressió oral bàsica i estructurada/
RA7: Diàlegs i converses senzilles/ RA8: Textos escrits breus & RA6:
Presentacions orals concretes/ RA7: Interacció en contextos personals i
laborals/ RA8: Textos escrits amb detall \\
\bottomrule
\end{longtable}

\begin{center}\rule{0.5\linewidth}{0.5pt}\end{center}

\hypertarget{distribuciuxf3-orientativa-de-responsabilitats-docents}{%
\subsubsection{7.4.4 Distribució orientativa de responsabilitats
docents}\label{distribuciuxf3-orientativa-de-responsabilitats-docents}}

\begin{longtable}[]{@{}lll@{}}
\toprule
\textbf{Àmbit} & \textbf{Responsable principal} & \textbf{Funcions
específiques} \\
\midrule
\endhead
Llengua Castellana/* & Professor/a de Llengua Castellana & Expressió
oral i escrita, anàlisi de textos, lectura literària, gramàtica,
ortografia \\
Ciències Socials/* & Professor/a de Geografia i Història & Evolució
històrica, ciutadania, democràcia, anàlisi de fonts i contextos
socials \\
Llengua Anglesa & Professor/a d'Anglés & Producció i comprensió oral i
escrita, vocabulari funcional, interacció en contextos reals \\
Coordinació general & Equip docent del cicle + cap d'estudis &
Planificació conjunta, seguiment d'alumnat, disseny de situacions
d'aprenentatge integrades \\
\bottomrule
\end{longtable}

\begin{quote}
/* \emph{Per motius d'organització interna i racionalització de recursos
docents, els àmbits de Llengua Castellana i Ciències Socials poden estar
integrats i ser atesos pel mateix professorat, habitualment el de
Llengua Castellana.}
\end{quote}

\hypertarget{criteris-generals-sobre-lorganitzaciuxf3-la-comunicaciuxf3-i-desenvolupament-del-procuxe9s-davaluaciuxf3-de-laprenentatge}{%
\section{8. Criteris generals sobre l/'organització, la comunicació i
desenvolupament del procés d/'avaluació de
l/'aprenentatge}\label{criteris-generals-sobre-lorganitzaciuxf3-la-comunicaciuxf3-i-desenvolupament-del-procuxe9s-davaluaciuxf3-de-laprenentatge}}

El procés d'avaluació de l'alumnat en el cicle de Formació Professional
Bàsica en Informàtica d'Oficina es regirà per:

\begin{itemize}
\tightlist
\item
  L'\textbf{Ordre 2025/13083} (DOGV 30.04.2025), que regula l'avaluació,
  acreditació, certificació i titulació en l'FP a la Comunitat
  Valenciana.
\item
  La \textbf{Llei orgànica 3/2022}, d'ordenació i integració de la FP.
\item
  El \textbf{Reial decret 659/2023}, pel qual s'estableix l'ordenació
  del sistema de Formació Professional.
\end{itemize}

Els principis generals que emanen d'aquesta normativa són:

\begin{enumerate}
\def\labelenumi{\arabic{enumi}.}
\tightlist
\item
  L'avaluació ha de ser \textbf{contínua, formativa i integradora},
  adaptada als ritmes i característiques de l'alumnat.
\item
  L'avaluació es basa en l'assoliment dels \textbf{Resultats
  d'Aprenentatge (RA)} definits per a cada mòdul.
\item
  El procés ha de facilitar l'\textbf{orientació, la millora dels
  aprenentatges} i la \textbf{inserció laboral} de l'alumnat.
\end{enumerate}

\hypertarget{organitzaciuxf3-del-procuxe9s-davaluaciuxf3}{%
\subsection{8.1 Organització del procés
d'avaluació}\label{organitzaciuxf3-del-procuxe9s-davaluaciuxf3}}

\begin{itemize}
\tightlist
\item
  Es realitzarà una \textbf{sessió inicial de diagnòstic} dins del
  primer mes de curs.
\item
  L'equip docent celebrarà \textbf{sessions d'avaluació trimestrals} i
  farà un seguiment setmanal de l'evolució de l'alumnat.
\item
  La \textbf{superació de cada mòdul} requerirà l'assoliment de tots els
  RA amb una qualificació mínima de 5 sobre 10.
\item
  L'alumnat amb una assistència inferior al 85\% podrà perdre el dret a
  l'avaluació contínua.
\item
  Es preveuen \textbf{convocatòries ordinària i extraordinària} per a la
  recuperació dels RA no assolits.
\item
  El Projecte Intermodular serà també objecte d'avaluació coordinada
  entre els docents implicats.
\end{itemize}

\hypertarget{instruments-davaluaciuxf3}{%
\subsection{8.2 Instruments
d'avaluació}\label{instruments-davaluaciuxf3}}

Els instruments d'avaluació es seleccionaran en funció del tipus de
tasca i metodologia utilitzada:

\begin{itemize}
\tightlist
\item
  \textbf{Proves pràctiques i escrites}
\item
  \textbf{Observació sistemàtica}
\item
  \textbf{Portafolis}
\item
  \textbf{Rúbriques i llistes de control}
\item
  \textbf{Autoavaluació i coavaluació}
\item
  \textbf{Seguiment de projectes i activitats col·laboratives}
\item
  \textbf{Presentacions orals i informes tècnics}
\end{itemize}

Els criteris d'avaluació s'ajustaran als RA i seran compartits amb
l'alumnat des de l'inici del curs.

\hypertarget{comunicaciuxf3-amb-lalumnat-i-les-famuxedlies}{%
\subsection{8.3 Comunicació amb l'alumnat i les
famílies}\label{comunicaciuxf3-amb-lalumnat-i-les-famuxedlies}}

\begin{itemize}
\tightlist
\item
  Els criteris i instruments d'avaluació es publicaran a través de la
  plataforma \emph{Aules} i seran explicats a l'alumnat a l'inici de
  cada mòdul.
\item
  L'equip docent mantindrà una \textbf{comunicació constant} mitjançant
  reunions de tutoria, notificacions digitals i entrevistes.
\item
  Els \textbf{butlletins oficials} de qualificacions es tramitaran a
  través de la plataforma \emph{ITACA}.
\item
  Es fomentarà la \textbf{transparència i la participació activa} de les
  famílies en el seguiment del procés educatiu.
\end{itemize}

\hypertarget{desenvolupament-del-procuxe9s-davaluaciuxf3}{%
\subsection{8.4 Desenvolupament del procés
d'avaluació}\label{desenvolupament-del-procuxe9s-davaluaciuxf3}}

\begin{itemize}
\tightlist
\item
  L'avaluació es realitzarà sobre \textbf{cada RA de forma
  diferenciada}.
\item
  Cada RA haurà d'estar \textbf{superat de forma individual} per
  considerar-se el mòdul aprovat.
\item
  El professorat farà \textbf{ajustaments metodològics i de
  planificació} segons l'evolució del grup.
\item
  L'avaluació continuarà en la \textbf{Formació en Entorns Laborals
  (FEE)}.
\end{itemize}

\hypertarget{avaluaciuxf3-en-la-formaciuxf3-en-entorns-laborals-fee}{%
\subsection{8.5 Avaluació en la Formació en Entorns Laborals
(FEE)}\label{avaluaciuxf3-en-la-formaciuxf3-en-entorns-laborals-fee}}

La FEE és una part fonamental del cicle formatiu. Es realitzarà:

\begin{itemize}
\tightlist
\item
  Mitjançant un \textbf{seguiment conjunt entre el centre educatiu i
  l'empresa}.
\item
  L'avaluació es fonamentarà en \textbf{l'assoliment dels RA relacionats
  amb el mòdul}, contextualitzats a l'entorn laboral.
\item
  L'empresa col·laboradora emetrà un \textbf{informe de valoració} i el
  tutor o tutora acadèmica completarà la qualificació.
\end{itemize}

\begin{quote}
\emph{L'avaluació de la FEE es regirà pels mateixos criteris de rigor,
objectivitat i coherència que la resta de mòduls del cicle.}
\end{quote}

\hypertarget{avaluaciuxf3-del-procuxe9s-docent-i-del-professorat}{%
\subsection{8.6 Avaluació del procés docent i del
professorat}\label{avaluaciuxf3-del-procuxe9s-docent-i-del-professorat}}

\begin{itemize}
\tightlist
\item
  El professorat realitzarà una \textbf{autoavaluació contínua} de la
  seua pràctica docent.
\item
  L'equip docent revisarà \textbf{trimestralment} la coherència i
  eficàcia de les seues programacions.
\item
  A final de curs, es farà una \textbf{avaluació col·lectiva} sobre:
\item
  La selecció i seqüenciació de continguts.
\item
  Els criteris i instruments d'avaluació utilitzats.
\item
  L'adequació de les metodologies.
\item
  Els resultats de l'alumnat i el seu progrés.
\end{itemize}

\hypertarget{base-de-dades-dempreses-o-organismes-equiparats-que-collaboren-amb-el-pccf-i-criteris-dassignaciuxf3-de-lalumnat}{%
\section{9. Base de dades d/'empreses o organismes equiparats que
col·laboren amb el PCCF i criteris d/'assignació de
l/'alumnat}\label{base-de-dades-dempreses-o-organismes-equiparats-que-collaboren-amb-el-pccf-i-criteris-dassignaciuxf3-de-lalumnat}}

La plataforma \textbf{SAO} per a la gestió de la FEE (Formació en
Entorns Laborals), així com de l'antiga FCT i la FP Dual, disposa d'una
\textbf{base de dades} amb el registre de les empreses que col·laboren
anualment amb el centre.

Tanmateix, l'equip educatiu ha acordat mantindre un \textbf{registre
digital complementari}, específic del departament, que reculla
l'historial de col·laboració amb les empreses i organismes equiparats.
Aquest registre podrà mantindre's en format \textbf{full de càlcul} o
integrat en la \textbf{plataforma web de la borsa de treball del
centre}.

Els responsables de l'actualització d'aquest registre seran el
\textbf{tutor o tutora del grup} i el/la \textbf{coordinador/a de
formació en empresa}, i es realitzarà \textbf{just després de la
finalització del període lectiu}. El registre inclourà, si és possible,
els \textbf{Resultats d'Aprenentatge (RA)} treballats per l'alumnat en
cada empresa, per tal d'ajustar els futurs \textbf{programes formatius
individuals}.

A més, es fomentarà la \textbf{col·laboració activa de les empreses} en:

\begin{itemize}
\tightlist
\item
  L'elaboració i revisió dels programes formatius.
\item
  El disseny i participació en projectes col·laboratius.
\item
  El desenvolupament d'una \textbf{xarxa de mentors} que done suport a
  l'alumnat en iniciatives d'emprenedoria.
\end{itemize}

\hypertarget{criteris-dassignaciuxf3-de-lalumnat-a-les-empreses}{%
\subsection{Criteris d'assignació de l'alumnat a les
empreses}\label{criteris-dassignaciuxf3-de-lalumnat-a-les-empreses}}

Per garantir una distribució equitativa i ajustada, es tindran en compte
els següents criteris:

\begin{itemize}
\tightlist
\item
  Adequació al perfil professional del cicle.
\item
  Competències, interessos i característiques personals de l'alumnat.
\item
  Ubicació geogràfica i possibilitat de desplaçament.
\item
  Capacitat de tutorització de l'empresa.
\item
  Garantia d'igualtat d'oportunitats.
\item
  Valoració de l'historial de col·laboració amb l'empresa.
\item
  Temporalització i disponibilitat per acollir alumnat.
\end{itemize}

És important destacar que el \textbf{teixit empresarial local} es compon
principalment de \textbf{petites i mitjanes empreses (pimes)}, i que les
empreses de major dimensió estan focalitzades en sectors específics dins
l'àmbit TIC.

Aquest context dificulta, sovint, trobar empreses que puguen acollir
alumnat en condicions òptimes per desenvolupar \textbf{tots els
Resultats d'Aprenentatge del cicle}. Per aquest motiu, i per garantir la
cobertura competencial, es preveu que els RA siguen \textbf{treballats
de manera complementària entre l'empresa i el centre educatiu}, Aquest
context dificulta, sovint, trobar empreses que puguen acollir alumnat en
condicions òptimes per desenvolupar \textbf{tots els Resultats
d'Aprenentatge del cicle}. Per aquest motiu, i per garantir la cobertura
competencial, es preveu que els RA siguen \textbf{treballats de manera
complementària entre l'empresa i el centre educatiu}.

La seua avaluació es basarà en \textbf{percentatges d'assoliment}, en
lloc de considerar els RA com a blocs tancats. Aquesta distribució
parcial i compartida haurà de quedar \textbf{clarament reflectida en les
programacions d'aula dels mòduls implicats}, indicant quines parts
s'abordaran a l'empresa i quines es desenvoluparan al centre, així com
els criteris i instruments d'avaluació associats.

\hypertarget{exemple-de-taula-de-distribuciuxf3-dels-ra-entre-centre-i-empresa}{%
\subsubsection{Exemple de taula de distribució dels RA entre centre i
empresa}\label{exemple-de-taula-de-distribuciuxf3-dels-ra-entre-centre-i-empresa}}

\begin{longtable}[]{@{}lllllll@{}}
\toprule
\textbf{Mòdul professional} & \textbf{Codi} & \textbf{RA} &
\textbf{Descripció del RA} & /\% al Centre & /\% a l/'Empresa &
Observacions \\
\midrule
\endhead
Aplicacions Ofimàtiques & 3013 & RA1 & Utilitza aplicacions de
processador de textos & 70\% & 30\% & Activitats de redacció a
l'empresa \\
Instal·lació i manteniment de xarxes & 3014 & RA2 & Realitza connexions
bàsiques i diagnòstics de xarxa & 50\% & 50\% & Pràctiques directes amb
client \\
Comunicació i Societat II & 3012 & RA6 & Comunica informació oral en
anglès en context laboral & 40\% & 60\% & Simulacions reals i atenció en
anglès \\
Tractament de la informació digital & 3015 & RA3 & Arxiva i gestiona
informació amb criteris de seguretat & 60\% & 40\% & Adaptat a protocols
de l'empresa \\
\bottomrule
\end{longtable}

\emph{Aquesta taula pot ser adaptada per a cada mòdul, i complementada
amb instruments d'avaluació específics (rúbriques, fulls d'observació,
informes de tutoria/\ldots).}

\hypertarget{criteris-per-a-realitzar-els-plans-formatius-individuals}{%
\section{10. Criteris per a realitzar els plans formatius
individuals}\label{criteris-per-a-realitzar-els-plans-formatius-individuals}}

Cada alumne o alumna que participe en un itinerari de \textbf{Formació
en Entorns Laborals (FEE)} haurà de disposar d'un \textbf{Pla Formatiu
Individualitzat (PFI)} que definisca de manera clara i flexible les
activitats formatives a desenvolupar tant al \textbf{centre educatiu}
com a l'\textbf{empresa o organisme col·laborador}.

Aquest Pla Formatiu serà el resultat d'un procés de \textbf{codisseny
entre el centre i l'empresa}, i haurà d'assegurar la \textbf{cobertura
de tots els Resultats d'Aprenentatge (RA)} establits en el currículum,
distribuint-los entre ambdós espais formatius segons la seua naturalesa,
les capacitats de l'empresa i les necessitats de l'alumnat.

\hypertarget{criteris-generals-per-a-la-seua-elaboraciuxf3}{%
\subsection{Criteris generals per a la seua
elaboració}\label{criteris-generals-per-a-la-seua-elaboraciuxf3}}

\begin{itemize}
\tightlist
\item
  El Pla Formatiu haurà de ser \textbf{coherent amb la programació
  d'aula del mòdul professional} i el \textbf{PAC del cicle}.
\item
  Es concretarà la \textbf{relació de RA assignats a l'empresa, al
  centre i compartits}, indicant els \textbf{percentatges d'assoliment
  estimats} en cada àmbit.
\item
  Caldrà indicar els \textbf{objectius específics}, \textbf{tasques
  formatives}, \textbf{condicions de seguiment}, així com els
  \textbf{instruments i criteris d'avaluació}.
\item
  S'inclouran les \textbf{competències per a l'ocupabilitat} a
  desenvolupar preferentment en context real.
\item
  Es preveuran \textbf{mecanismes de revisió i ajust del pla},
  especialment quan es produïsquen canvis rellevants en el context
  formatiu o empresarial.
\end{itemize}

\hypertarget{documentaciuxf3-i-protocol}{%
\subsection{Documentació i protocol}\label{documentaciuxf3-i-protocol}}

Per garantir l'homogeneïtat i el seguiment de qualitat, el centre
establirà un \textbf{model de document de PFI}, que inclourà com a
mínim:

\begin{itemize}
\tightlist
\item
  Dades de l'alumne/a i de l'empresa col·laboradora.
\item
  Responsable de tutoria acadèmica i de supervisió a l'empresa.
\item
  RA assignats i percentatges corresponents.
\item
  Activitats formatives previstes.
\item
  Indicadors d'avaluació i instruments associats.
\item
  Dates clau i horari previst.
\item
  Signatura de les parts implicades.
\end{itemize}

Aquest document serà \textbf{signat pel centre, l'empresa i l'alumne/a},
i constituirà el marc de referència per al seguiment i l'avaluació de
l'estada formativa.

Els Plans Formatius Individuals s'arxivaran en format digital i estaran
\textbf{disponibles per a la inspecció educativa, la coordinació de FEE
i l'equip docent}.

\begin{quote}
🗂 \emph{A més, la informació continguda en el PFI podrà ser incorporada
com a evidència dins del portafoli d'avaluació de l'alumnat.}
\end{quote}

\hypertarget{criteris-per-a-adaptar-els-muxf2duls-de-digitalitzaciuxf3-i-sostenibilitat-a-les-caracteruxedstiques-especuxedfiques-del-perfil-professional-del-cicle-formatiu}{%
\section{11. Criteris per a adaptar els mòduls de Digitalització i
Sostenibilitat a les característiques específiques del perfil
professional del cicle
formatiu}\label{criteris-per-a-adaptar-els-muxf2duls-de-digitalitzaciuxf3-i-sostenibilitat-a-les-caracteruxedstiques-especuxedfiques-del-perfil-professional-del-cicle-formatiu}}

Els mòduls \textbf{transversals} de \textbf{Digitalització Aplicada als
Sectors Productius} i \textbf{Sostenibilitat Aplicada al Sistema
Productiu} formen part del currículum comú de la nova Formació
Professional i tenen un \textbf{caràcter integrador i contextual}. Per
això, és necessari \textbf{adaptar-ne els continguts, activitats i
enfocaments} a les \textbf{característiques del cicle formatiu de FPB en
Informàtica d'Oficina}, així com al \textbf{perfil professional de
l'alumnat} i a la realitat del teixit productiu de l/'entorn.

\hypertarget{adaptaciuxf3-del-muxf2dul-de-digitalitzaciuxf3}{%
\subsection{Adaptació del mòdul de
Digitalització}\label{adaptaciuxf3-del-muxf2dul-de-digitalitzaciuxf3}}

El mòdul de \textbf{Digitalització} s'orientarà a dotar l'alumnat de
competències bàsiques per a comprendre i aplicar processos de
digitalització en contextos administratius i d'oficina. En concret, es
treballaran:

\begin{itemize}
\tightlist
\item
  \textbf{Ús d'eines digitals col·laboratives} (drive, calendari, suite
  ofimàtica en núvol, etc.)
\item
  \textbf{Digitalització de documents i arxius}, amb formats
  estandarditzats i criteris d'organització segura.
\item
  \textbf{Coneixements bàsics de ciberseguretat i protecció de dades},
  aplicats a la gestió documental.
\item
  Introducció al concepte d'\textbf{automatització de tasques
  administratives} amb programari d'oficina.
\item
  Participació en \textbf{projectes de digitalització del centre o de la
  comunitat}, com per exemple, la gestió de fitxers o l'organització de
  dades per a esdeveniments escolars.
\end{itemize}

\hypertarget{adaptaciuxf3-del-muxf2dul-de-sostenibilitat}{%
\subsection{Adaptació del mòdul de
Sostenibilitat}\label{adaptaciuxf3-del-muxf2dul-de-sostenibilitat}}

Pel que fa al mòdul de \textbf{Sostenibilitat}, es proposarà una
aproximació pràctica i contextualitzada, que ajude a l'alumnat a
comprendre l'impacte ambiental i social de les activitats
administratives i digitals. Es prioritzaran:

\begin{itemize}
\tightlist
\item
  Pràctiques d'\textbf{estalvi de recursos} en l'ús de tecnologies
  (paper, tinta, energia/\ldots).
\item
  \textbf{Gestió responsable de residus electrònics} (tòners, aparells
  informàtics obsolets).
\item
  Foment del \textbf{consum responsable}, especialment en compres i
  equipaments per a l'oficina.
\item
  Introducció al càlcul de la \textbf{petjada ecològica} i a l'impacte
  dels hàbits digitals (ús de núvol, enviament massiu de correus, etc.).
\item
  Participació en \textbf{projectes transversals} com el projecte de
  \textbf{reciclatge de plàstic i impressió 3D}, en col·laboració amb
  altres mòduls.
\end{itemize}

\hypertarget{enfocament-transversal-i-coordinat}{%
\subsection{Enfocament transversal i
coordinat}\label{enfocament-transversal-i-coordinat}}

Ambdós mòduls es desplegaran en coordinació amb la resta de mòduls del
cicle, i \textbf{preferentment dins de situacions d'aprenentatge o
projectes intermodulars}. Això garantirà:

\begin{itemize}
\tightlist
\item
  \textbf{Contextualització real dels continguts}, afavorint la
  transferència d'aprenentatges.
\item
  \textbf{Avaluació integrada} a través de projectes i activitats
  col·laboratives.
\item
  \textbf{Enfocament competencial i vivencial}, orientat a la inserció
  laboral i la ciutadania responsable.
\end{itemize}

L'equip docent establirà anualment, dins de la \textbf{PGA}, quines
activitats o projectes oferiran millor encaix per integrar
\textbf{digitalització i sostenibilitat} dins del cicle, i quins
criteris s'utilitzaran per a la seua avaluació.

\hypertarget{el-pla-de-tutoria-i-orientaciuxf3-professional}{%
\section{12. El pla de tutoria i orientació
professional}\label{el-pla-de-tutoria-i-orientaciuxf3-professional}}

Per al cicle formatiu de \textbf{Formació Professional Bàsica en
Informàtica d'Oficina}, i d'acord amb l'article 91 de la LOMLOE, el
Decret 72/2021 i l'Orde 10/2023, l'equip docent estableix les següents
\textbf{línies estratègiques de tutoria i orientació professional}.
Aquestes s'han de concretar en els \textbf{plans d'acció tutorial (PAT)}
anuals de cada grup.

\hypertarget{organitzaciuxf3-de-la-tutoria}{%
\subsection{1. Organització de la
tutoria}\label{organitzaciuxf3-de-la-tutoria}}

\begin{itemize}
\tightlist
\item
  El/la tutor/a serà un/a docent del grup amb hores assignades a tutoria
  segons la PGA.
\item
  Encara que no hi haja hora setmanal específica, es podran fer
  \textbf{sessions de tutoria dins de les hores lectives} del mòdul
  assignat al/la tutor/a.
\item
  Les \textbf{sessions grupals} es dedicaran a aspectes personals,
  acadèmics i professionals.
\item
  Les \textbf{entrevistes individuals} serviran per fer seguiment i
  donar suport a l'alumnat amb NESE o situacions personals complexes.
\item
  Es garantirà la coordinació amb el \textbf{departament d'orientació}
  per:
\item
  Detectar necessitats específiques.
\item
  Establir estratègies individualitzades.
\item
  Fer acompanyament emocional i acadèmic.
\end{itemize}

\hypertarget{orientaciuxf3-acaduxe8mica}{%
\subsection{2. Orientació acadèmica}\label{orientaciuxf3-acaduxe8mica}}

\begin{itemize}
\tightlist
\item
  S'organitzaran activitats per a informar sobre opcions formatives
  posteriors:
\item
  Cicles de Grau Mitjà de la mateixa família professional.
\item
  Cursos d'especialització i certificacions professionals.
\item
  Altres vies formatives (ocupacional, PICE, formació a distància,
  etc.).
\item
  Es realitzaran:
\item
  \textbf{Xarrades amb professorat} de cicles superiors.
\item
  \textbf{Trobades amb exalumnes} titulats.
\item
  \textbf{Sessions informatives} sobre les comissions col·legiades
  d'orientació.
\end{itemize}

\hypertarget{orientaciuxf3-professional-inserciuxf3-laboral-i-emprenedoria}{%
\subsection{3. Orientació professional, inserció laboral i
emprenedoria}\label{orientaciuxf3-professional-inserciuxf3-laboral-i-emprenedoria}}

\begin{itemize}
\tightlist
\item
  L'alumnat participarà en activitats com:
\item
  \textbf{Taller de recerca d'ocupació} (CV, entrevistes, portals).
\item
  \textbf{Fires d'ocupació i visites a empreses}.
\item
  \textbf{Xarrades amb professionals del sector}.
\item
  Difusió del \textbf{caràcter dual de la FP} i possibilitats d'inserció
  directa.
\item
  \textbf{Hackatons, concursos i activitats de captació de talent}.
\item
  Mentories amb professionals i exalumnes per fomentar l'emprenedoria.
\item
  Orientació cap a \textbf{reptes reals} que puguen convertir-se en
  projectes d'empresa.
\end{itemize}

\hypertarget{seguiment-i-avaluaciuxf3-del-pla}{%
\subsection{4. Seguiment i avaluació del
pla}\label{seguiment-i-avaluaciuxf3-del-pla}}

\begin{itemize}
\tightlist
\item
  El Pla de Tutoria i Orientació es \textbf{avaluarà anualment}, durant
  el tercer trimestre, amb:
\item
  \textbf{Enquestes de valoració de l'alumnat}.
\item
  \textbf{Reunions de l'equip docent i amb el departament d'orientació}.
\item
  \textbf{Informe final de valoració del PAT}, amb propostes de millora.
\item
  Els resultats d'aquesta avaluació es tindran en compte per ajustar les
  accions del curs següent.
\end{itemize}

\begin{center}\rule{0.5\linewidth}{0.5pt}\end{center}

\hypertarget{clip-annexos-del-pla-de-tutoria-i-orientaciuxf3}{%
\subsection{/{[}clip/{]} Annexos del pla de tutoria i
orientació}\label{clip-annexos-del-pla-de-tutoria-i-orientaciuxf3}}

\hypertarget{annex-1.-model-dentrevista-individual}{%
\subsubsection{\texorpdfstring{\href{../annexos/Annex1-Entrevista/}{Annex
1. Model d'entrevista
individual}}{Annex 1. Model d'entrevista individual}}\label{annex-1.-model-dentrevista-individual}}

\hypertarget{annex-2.-quxfcestionari-dinteressos-professionals}{%
\subsubsection{\texorpdfstring{\href{../annexos/Annex2-Interessos/}{Annex
2. Qüestionari d'interessos
professionals}}{Annex 2. Qüestionari d'interessos professionals}}\label{annex-2.-quxfcestionari-dinteressos-professionals}}

\hypertarget{annex-3.-fitxa-de-seguiment-del-pat}{%
\subsubsection{\texorpdfstring{\href{../annexos/Annex3-SeguimentPAT/}{Annex
3. Fitxa de seguiment del
PAT}}{Annex 3. Fitxa de seguiment del PAT}}\label{annex-3.-fitxa-de-seguiment-del-pat}}

\hypertarget{annex-4.-enquesta-de-valoraciuxf3-del-pla-alumnat}{%
\subsubsection{\texorpdfstring{\href{../annexos/Annex4-EnquestaAlumnat/}{Annex
4. Enquesta de valoració del pla
(alumnat)}}{Annex 4. Enquesta de valoració del pla (alumnat)}}\label{annex-4.-enquesta-de-valoraciuxf3-del-pla-alumnat}}

\hypertarget{concreciuxf3-dels-plans-i-els-programes-del-centre-vinculats-al-curruxedculum}{%
\section{13. Concreció dels plans i els programes del centre vinculats
al
currículum}\label{concreciuxf3-dels-plans-i-els-programes-del-centre-vinculats-al-curruxedculum}}

Aquest apartat concreta com es despleguen, dins del cicle de
\textbf{Formació Professional Bàsica en Informàtica d'Oficina}, els
\textbf{plans, projectes i programes impulsats pel centre}, assegurant
la seua \textbf{coherència amb els objectius del currículum} i la
formació integral de l'alumnat. Es descriuen accions específiques,
mòduls implicats, responsables i temporalització aproximada.

\hypertarget{programa-empruxe9n-aula-emprenedora}{%
\subsection{1. Programa Emprén -- Aula
Emprenedora}\label{programa-empruxe9n-aula-emprenedora}}

El centre participa activament en el \textbf{Programa Emprén}, de foment
de la cultura emprenedora. L'Aula Emprenedora promou:

\begin{itemize}
\tightlist
\item
  L'\textbf{emprenedoria social, sostenible i col·laborativa}.
\item
  El desenvolupament d'\textbf{habilitats personals i professionals}.
\item
  La sensibilització cap a models de negoci responsables i innovadors.
\end{itemize}

\hypertarget{projecte-justhub-garage-espai-dinnovaciuxf3-i-creativitat}{%
\subsection{2. Projecte /``JustHub Garage -- Espai d'Innovació i
Creativitat/''}\label{projecte-justhub-garage-espai-dinnovaciuxf3-i-creativitat}}

Llançat el curs 2024, aquest projecte transforma l'aula en un
\textbf{entorn de coworking i laboratori de prototipatge}, inspirat en
les empreses nascudes en garatges. Està alineat amb l'estratègia
d'\textbf{Aules Transformadores}, i s'hi desenvolupen:

\begin{itemize}
\tightlist
\item
  Projectes interdisciplinars basats en reptes reals del sector TIC.
\item
  Metodologies com \textbf{Design Thinking}, \textbf{Scrum} o
  \textbf{ABP}.
\item
  Espais que simulen una \textbf{aula-empresa col·laborativa}.
\end{itemize}

\hypertarget{accions-concretes}{%
\subsubsection{Accions concretes:}\label{accions-concretes}}

\begin{longtable}[]{@{}llll@{}}
\toprule
Acció & Mòduls implicats & Temporalització & Responsables \\
\midrule
\endhead
Desenvolupament de projectes col·laboratius (coworking) & IMX, CAII,
Comunicació i Societat II & Octubre - Maig & Equip docent i mentors
externs \\
Introducció a metodologies àgils & IMX, Sostenibilitat & Octubre - Gener
& Coordinador/a JustHub \\
Projectes de sostenibilitat i digitalització & Tots els mòduls & Tot el
curs & Coordinació docent \\
Tallers i xarrades d'emprenedoria & Tutoria, Comunicació & Novembre -
Abril & Aula Emprén + empreses col·laboradores \\
Mostra final de projectes (demo day) & Tots els mòduls & Final de curs &
Equip docent i col·laboradors externs \\
\bottomrule
\end{longtable}

\hypertarget{altres-plans-i-programes}{%
\subsection{3. Altres plans i
programes}\label{altres-plans-i-programes}}

\begin{itemize}
\tightlist
\item
  \textbf{Orientació Professional (Decret 72/2021)}: Integrada en les
  tutories i activitats del mòdul de Comunicació, coordinada amb el
  Departament d'Orientació.
\item
  \textbf{Formació i inserció laboral (FCT i borsa d'ocupació)}:
  Coordinació amb empreses locals i serveis municipals d'ocupació.
\item
  \textbf{Participació en convocatòries INNOVATEC o projectes de
  millora}: Quan siga aplicable, els projectes d'innovació es
  connectaran amb continguts del mòdul de digitalització o
  sostenibilitat.
\end{itemize}

\hypertarget{objectius-diduxe0ctics-i-alineaciuxf3-curricular}{%
\subsection{4. Objectius didàctics i alineació
curricular}\label{objectius-diduxe0ctics-i-alineaciuxf3-curricular}}

Tots aquests programes reforcen:

\begin{itemize}
\tightlist
\item
  El treball de les \textbf{competències professionals, personals i
  socials}.
\item
  La connexió amb els \textbf{ODS} (Agenda 2030), especialment en
  l'àmbit tecnològic i ambiental.
\item
  L'adquisició d'\textbf{experiència pràctica i projectes significatius}
  per a l'alumnat.
\end{itemize}

Amb aquesta visió transversal, es fomenta una formació contextualitzada,
flexible i orientada a la ciutadania activa i emprenedora.

\hypertarget{orientacions-per-a-luxfas-despais-mitjans-i-equipaments-disponibles}{%
\section{14. Orientacions per a l/'ús d/'espais, mitjans i equipaments
disponibles}\label{orientacions-per-a-luxfas-despais-mitjans-i-equipaments-disponibles}}

L'organització dels espais i dels recursos materials en el cicle de
\textbf{Formació Professional Bàsica en Informàtica d'Oficina} té com a
objectiu \textbf{facilitar un aprenentatge actiu, segur i
contextualitzat}, d'acord amb els requisits establerts pel \emph{Reial
decret 405/2023} i alineat amb els principis d'innovació metodològica i
pedagògica.

\hypertarget{condicions-dels-espais-docents} del
  temps disponible.
\item
  Es garantiran les següents \textbf{condicions ambientals i de
  seguretat}:
\item
  \textbf{Humitat}: controlada per evitar condensació i electricitat
  estàtica.
\item
  \textbf{Il·luminació}: lateral i difusa, per evitar reflexos en
  pantalles.
\item
  \textbf{Cablejat}: centralitzat, segur i fora de zones de pas.
\item
  \textbf{Mobiliari ergonòmic}: per protegir la salut postural de
  l'alumnat.
\end{itemize}

\hypertarget{transformaciuxf3-metodoluxf2gica-i-aula-empresa}{%
\subsection{2. Transformació metodològica i
Aula-Empresa}\label{transformaciuxf3-metodoluxf2gica-i-aula-empresa}}

El cicle es desplega seguint un model d'\textbf{aula-empresa}, on l'aula
simula un \textbf{entorn productiu col·laboratiu}. Aquesta transformació
implica:

\begin{itemize}
\tightlist
\item
  Organització de l'espai en \textbf{illes de treball col·laboratiu}.
\item
  Creació d'una \textbf{zona de reunions} per a planificació i feedback.
\item
  Utilització de \textbf{mobiliari flexible} (pissarres mòbils,
  pantalles compartides).
\item
  Espais diferenciats per a \textbf{presentació de projectes} o defensa
  d'evidències.
\end{itemize}

Aquest entorn afavoreix metodologies com l'\textbf{Aprenentatge Basat en
Projectes (ABP)}, \textbf{reptes}, o \textbf{simulacions d'entorns
laborals}, fonamentals per a l'adquisició de les competències del cicle.

\hypertarget{uxfas-de-laula-empruxe9n}{%
\subsection{3. Ús de l'Aula Emprén}\label{uxfas-de-laula-empruxe9n}}

El centre disposa d'una \textbf{Aula Emprén}, vinculada al Programa
Emprén. Aquesta aula permet desenvolupar:

\begin{itemize}
\tightlist
\item
  \textbf{Tallers de creativitat i ideació de projectes}
\item
  \textbf{Hackatons i concursos} interdisciplinars
\item
  \textbf{Xarrades amb empreses i mentors}
\item
  \textbf{Simulació de presentacions professionals} i esdeveniments
\end{itemize}

S'utilitzarà especialment per a l'impuls del projecte \textbf{JustHub
Garage} i per activitats dels mòduls de \emph{Tutoria},
\emph{Comunicació i Societat} i \emph{Sostenibilitat}.

\hypertarget{coordinaciuxf3-i-uxfas-compartit-dels-espais}{%
\subsection{4. Coordinació i ús compartit dels
espais}\label{coordinaciuxf3-i-uxfas-compartit-dels-espais}}

Per optimitzar els espais del centre i garantir-ne un ús eficient:

\begin{itemize}
\tightlist
\item
  Es coordinarà la utilització de les aules especialitzades
  (\emph{tècniques, informàtiques, Emprén}) entre els diferents mòduls i
  grups.
\item
  Es podran establir \textbf{protocols d'ús i reserva}.
\item
  Es prioritzaran aquelles activitats relacionades amb \textbf{projectes
  intermodulars, digitalització i sostenibilitat}.
\end{itemize}

\begin{center}\rule{0.5\linewidth}{0.5pt}\end{center}

\textbf{Conclusió}:/ L/'ús intencionat dels espais i equipaments
disponibles permetrà \textbf{contextualitzar l'aprenentatge}, millorar
la \textbf{motivació de l'alumnat}, i connectar-lo amb els
\textbf{entorns reals de treball}, facilitant així l'assoliment dels
resultats d'aprenentatge establerts.

\hypertarget{criteris-i-procediments-per-a-lavaluaciuxf3-i-la-revisiuxf3-de-la-pruxe0ctica-docent}{%
\section{15. Criteris i procediments per a l/'avaluació i la revisió de
la pràctica
docent}\label{criteris-i-procediments-per-a-lavaluaciuxf3-i-la-revisiuxf3-de-la-pruxe0ctica-docent}}

L'avaluació i revisió de la \textbf{pràctica docent} en la Formació
Professional Bàsica és una eina clau per garantir la qualitat educativa
i millorar de manera contínua el procés d'ensenyament-aprenentatge. En
aquesta etapa, on l'alumnat sovint presenta trajectòries escolars
complexes, cal una revisió docent \textbf{especialment sensible a la
motivació, inclusió i orientació professional}.

Aquest procés d'avaluació es fonamenta en els \textbf{principis de la
LOMLOE} (article 1, LO 3/2020) i inclou la reflexió sobre la
\textbf{planificació, execució i resultats}, tant individuals com
col·lectius, de la tasca docent.

\hypertarget{criteris-per-a-la-revisiuxf3-de-la-pruxe0ctica-docent}{%
\subsection{Criteris per a la revisió de la pràctica
docent}\label{criteris-per-a-la-revisiuxf3-de-la-pruxe0ctica-docent}}

\begin{enumerate}
\def\labelenumi{\arabic{enumi}.}
\item
  \textbf{Adequació al currículum de FPB}:

  \begin{itemize}
  \tightlist
  \item
    Ajust de les programacions als resultats d'aprenentatge i sabers
    bàsics del currículum.
  \item
    Adaptació dels continguts al nivell competencial i ritme de
    l'alumnat.
  \item
    Coherència entre sabers, criteris d'avaluació i instruments
    aplicats.
  \end{itemize}
\item
  \textbf{Adaptació a les característiques de l'alumnat}:

  \begin{itemize}
  \tightlist
  \item
    Aplicació de mesures universals, addicionals o personalitzades per
    garantir l'èxit educatiu.
  \item
    Consideració del pla personal d'aprenentatge (PPA) quan siga
    aplicable.
  \end{itemize}
\item
  \textbf{Metodologies actives i personalitzades}:

  \begin{itemize}
  \tightlist
  \item
    Ús d'ABP, APS, reptes, tallers i simulacions com a estratègies
    habituals.
  \item
    Participació de l'alumnat com a protagonista del seu procés
    d'aprenentatge.
  \end{itemize}
\item
  \textbf{Inclusió i gestió d'aula}:

  \begin{itemize}
  \tightlist
  \item
    Promoció d'un clima positiu, respectuós i segur.
  \item
    Ús d'estratègies de gestió emocional, resolució de conflictes i
    mediació educativa.
  \end{itemize}
\item
  \textbf{Avaluació formativa i competencial}:

  \begin{itemize}
  \tightlist
  \item
    Aplicació de rúbriques, escales descriptives i portafolis.
  \item
    Observació del progrés i ajust del procés d'ensenyament en temps
    real.
  \end{itemize}
\item
  \textbf{Orientació i acompanyament}:

  \begin{itemize}
  \tightlist
  \item
    Integració del pla de tutoria i orientació dins de les activitats
    del mòdul.
  \item
    Coordinació amb el departament d'orientació i serveis socials si
    cal.
  \end{itemize}
\end{enumerate}

\hypertarget{procediments-davaluaciuxf3-de-la-pruxe0ctica-docent}{%
\subsection{Procediments d'avaluació de la pràctica
docent}\label{procediments-davaluaciuxf3-de-la-pruxe0ctica-docent}}

\begin{itemize}
\item
  \textbf{Autoavaluació docent}: cada professor realitzarà una reflexió
  individual (almenys un cop per curs) sobre les seues fortaleses, àrees
  de millora i necessitats de suport.
\item
  \textbf{Anàlisi col·lectiva}: l'equip docent realitzarà una reunió de
  revisió trimestral per compartir bones pràctiques, revisar dificultats
  i proposar millores.
\item
  \textbf{Enquestes de valoració de l'alumnat}: es passarà una enquesta
  breu a final de cada trimestre o quadrimestre per recollir la seua
  percepció sobre el clima d'aula, les activitats i el seu propi
  aprenentatge.
\item
  \textbf{Valoració conjunta amb l'equip orientador}: especialment en
  casos amb PPA, es farà una valoració específica de les adaptacions
  aplicades i de l'impacte en la motivació i l'assistència de l'alumnat.
\end{itemize}

\hypertarget{revisiuxf3-del-projecte-curricular-del-cicle-formatiu-pccf}{%
\subsection{Revisió del Projecte Curricular del Cicle Formatiu
(PCCF)}\label{revisiuxf3-del-projecte-curricular-del-cicle-formatiu-pccf}}

El \textbf{PCCF s'analitzarà anualment} al final del curs escolar en una
sessió d'avaluació global. En aquest espai, es valorarà:

\begin{itemize}
\tightlist
\item
  La implementació real dels criteris i metodologies acordades.
\item
  L'adequació dels recursos, espais i materials.
\item
  L'impacte dels projectes transversals en l'assoliment de competències.
\item
  La coordinació entre mòduls i amb serveis externs.
\end{itemize}

El resultat d'aquesta revisió podrà comportar \textbf{modificacions
concretes} per al següent curs, que quedaran recollides com a
\emph{actualitzacions} del PCCF, sempre d'acord amb el principi de
millora contínua i col·laborativa.

\begin{center}\rule{0.5\linewidth}{0.5pt}\end{center}

\hypertarget{clip-annexos-de-suport}{%
\subsection{/{[}clip/{]} Annexos de
suport}\label{clip-annexos-de-suport}}

Per facilitar el procés de revisió i reflexió docent, es proposen els
següents annexos:

\begin{itemize}
\tightlist
\item
  \href{../annexos/Annex-Autoavaluacio-Docent/}{/{[}nota/{]} Model
  d'autoavaluació docent}
\item
  \href{../annexos/Annex-Enquesta-Alumnat/}{/{[}llista/{]} Enquesta de
  valoració del mòdul (alumnat)}
\end{itemize}

Aquests materials es poden adaptar a cada mòdul o departament, i poden
formar part de la documentació habitual de final de trimestre o de curs.

\hypertarget{atenciuxf3-a-la-diversitat.-mesures-de-resposta-educativa-a-la-inclusiuxf3}{%
\section{16. Atenció a la diversitat. Mesures de resposta educativa a la
inclusió}\label{atenciuxf3-a-la-diversitat.-mesures-de-resposta-educativa-a-la-inclusiuxf3}}

L/'atenció a les diferències individuals és un \textbf{precepte
constitucional} (Constitució Espanyola, Art. 27.1) i un pilar de
l/'educació inclusiva. En l/'àmbit de la Formació Professional Bàsica
(FPB), aquest compromís es concreta mitjançant les \textbf{Mesures de
Resposta Educativa per a la Inclusió (MREI)}, tal com estableixen la
\textbf{LO 3/2022} i el \textbf{RD 659/2023}.

Aquestes mesures tenen com a finalitat \textbf{eliminar les barreres}
d/'accés, participació i aprenentatge, i assegurar que tot l'alumnat
tinga oportunitats reals per assolir les competències del cicle.

\hypertarget{principis-normatius}{%
\subsection{Principis normatius}\label{principis-normatius}}

D'acord amb el RD 659/2023, article 15, les mesures s'han d'ajustar a:

\begin{itemize}
\tightlist
\item
  \textbf{Normalització i inclusió educativa}
\item
  \textbf{Accessibilitat universal i disseny per a tots}
\item
  \textbf{Adaptació de condicions d'aprenentatge i avaluació}
\end{itemize}

\hypertarget{tipologia-de-mesures-mrei}{%
\subsection{Tipologia de mesures
(MREI)}\label{tipologia-de-mesures-mrei}}

\hypertarget{adaptacions-metodoluxf2giques}{%
\subsubsection{1. Adaptacions
metodològiques}\label{adaptacions-metodoluxf2giques}}

\begin{itemize}
\tightlist
\item
  Ús de \textbf{metodologies actives i flexibles} (ABP, reptes, tallers
  col·laboratius).
\item
  Disseny d'\textbf{activitats graduades} que permeten diferents nivells
  d/'assoliment.
\item
  Atenció a la diversitat d'estils d'aprenentatge (visual, kinestèsic,
  etc.).
\end{itemize}

\hypertarget{suport-personalitzat-i-tutoritzaciuxf3}{%
\subsubsection{2. Suport personalitzat i
tutorització}\label{suport-personalitzat-i-tutoritzaciuxf3}}

\begin{itemize}
\tightlist
\item
  \textbf{Tutories individualitzades} per establir plans de seguiment o
  PPA.
\item
  Dinàmiques de suport emocional, regulació i motivació.
\item
  Observació sistemàtica per identificar necessitats emergents.
\end{itemize}

\hypertarget{adaptacions-dorganitzaciuxf3-i-despais}{%
\subsubsection{3. Adaptacions d'organització i
d'espais}\label{adaptacions-dorganitzaciuxf3-i-despais}}

\begin{itemize}
\tightlist
\item
  Agrupaments flexibles i suport dins l'aula ordinària.
\item
  Ajustos en la distribució de temps, activitats i materials.
\item
  Coordinació amb el departament d'orientació i serveis externs.
\end{itemize}

\hypertarget{accessibilitat-i-recursos-adaptats}{%
\subsubsection{4. Accessibilitat i recursos
adaptats}\label{accessibilitat-i-recursos-adaptats}}

\begin{itemize}
\tightlist
\item
  Materials accessibles (ampliació de tipografia, contrast, lectura
  fàcil/\ldots).
\item
  Ús de \textbf{tecnologies assistives} (lector de pantalla, subtítols,
  teclats especials/\ldots).
\item
  Eliminació de barreres físiques, sensorials i digitals.
\end{itemize}

\hypertarget{avaluaciuxf3-inclusiva}{%
\subsection{Avaluació inclusiva}\label{avaluaciuxf3-inclusiva}}

\begin{itemize}
\tightlist
\item
  Avaluació \textbf{personalitzada i formativa}, orientada al progrés.
\item
  Ús d'\textbf{instruments diversos i adaptats} (rúbriques, observació,
  exposicions, autoavaluació).
\item
  Flexibilitat en les condicions (temps extra, format de proves,
  presentacions orals/\ldots).
\end{itemize}

\hypertarget{revisiuxf3-i-seguiment}{%
\subsection{Revisió i seguiment}\label{revisiuxf3-i-seguiment}}

\begin{itemize}
\tightlist
\item
  Les MREI seran acordades i revisades \textbf{col·lectivament} per
  l'equip docent en coordinació amb el departament d'orientació.
\item
  Les mesures aplicades es revisaran \textbf{trimestralment}, i
  s'ajustaran segons l'evolució de l'alumnat i les necessitats
  detectades.
\end{itemize}

\hypertarget{referuxe8ncies-normatives-i-recursos}{%
\subsection{Referències normatives i
recursos}\label{referuxe8ncies-normatives-i-recursos}}

\begin{itemize}
\tightlist
\item
  LO 3/2022, d/'ordenació i integració de la Formació Professional
\item
  RD 659/2023, d'ordenació dels cicles de grau bàsic
\item
  GVA: \href{https://ceice.gva.es/va/web/inclusioeducativa}{Mesures
  d'inclusió educativa}
\item
  GVA:
  \href{https://ceice.gva.es/es/web/inclusioeducativa/identificacio-de-barreres}{Identificació
  de barreres}
\end{itemize}

\hypertarget{comentari-final}{%
\subsection{Comentari final}\label{comentari-final}}

En el marc del Projecte Curricular de Cicle Formatiu, les MREI
s'integren com a part fonamental del model pedagògic del centre. La seua
aplicació en FPB esdevé imprescindible per garantir l'\textbf{equitat
real} i la \textbf{justícia educativa}, superant una visió només
compensatòria.

\begin{quote}
/{[}idea/{]} Les MREI no són excepcions, sinó respostes sistemàtiques i
estructurades que reconeixen i abracen la diversitat.
\end{quote}

\hypertarget{criteris-per-a-la-planificaciuxf3-dactivitats-complementuxe0ries-i-extraescolars}{%
\section{17. Criteris per a la planificació d/'activitats
complementàries i
extraescolars}\label{criteris-per-a-la-planificaciuxf3-dactivitats-complementuxe0ries-i-extraescolars}}

Les \textbf{activitats complementàries i extraescolars} són eines
pedagògiques que \textbf{enriqueixen i consoliden} el procés
d'ensenyament-aprenentatge. Permeten a l'alumnat aplicar els
coneixements adquirits en entorns reals, millorar la motivació i
desenvolupar competències transversals.

Aquestes activitats es programaran seguint els criteris acordats per
l'equip docent i alineats amb els \textbf{objectius del currículum del
cicle FPB d'Informàtica d'Oficina}.

\hypertarget{criteris-generals-per-a-la-planificaciuxf3}{%
\subsection{Criteris generals per a la
planificació}\label{criteris-generals-per-a-la-planificaciuxf3}}

\begin{enumerate}
\def\labelenumi{\arabic{enumi}.}
\tightlist
\item
  \textbf{Pertinència curricular}: han de reforçar els continguts i
  sabers treballats als mòduls.
\item
  \textbf{Caràcter inclusiu}: totes les activitats han de ser
  accessibles i promoure la participació de tot l'alumnat.
\item
  \textbf{Coherència amb la programació d/'aula}: han d'estar
  relacionades amb els objectius i competències específiques dels
  mòduls.
\item
  \textbf{Viabilitat organitzativa}: es valorarà la logística, horaris,
  pressupost i recursos disponibles.
\item
  \textbf{Informació prèvia a l'alumnat i famílies}: es facilitarà una
  comunicació clara sobre objectius, continguts, dates i requisits de
  participació.
\end{enumerate}

\hypertarget{tipus-dactivitats}{%
\subsection{Tipus d/'activitats}\label{tipus-dactivitats}}

\hypertarget{activitats-complementuxe0ries}{%
\subsubsection{Activitats
complementàries}\label{activitats-complementuxe0ries}}

Realitzades dins de l'horari lectiu, amb caràcter pedagògic i
estretament vinculades al currículum.

\begin{itemize}
\tightlist
\item
  \textbf{Tallers tecnològics} (automatització, IA, impressió 3D\ldots)
\item
  \textbf{Jornada de contacte amb empreses del sector}
\item
  \textbf{Seminaris d'especialització} amb professionals TIC
\item
  \textbf{Visita a la Fossa 112 de Paterna} (memòria històrica i
  tecnologia aplicada)
\item
  \textbf{Eixides de curta durada a l'entorn de la ciutat i muntanya}
  (orientació, treball en equip, medi ambient)
\item
  \textbf{Hackatons i jams tecnològiques} dins del centre
\end{itemize}

\hypertarget{activitats-extraescolars}{%
\subsubsection{Activitats
extraescolars}\label{activitats-extraescolars}}

Fora de l'horari lectiu, obertes a la participació voluntària i
orientades a l'enriquiment acadèmic i personal.

\begin{itemize}
\tightlist
\item
  \textbf{Visita al Museu del videojoc d'Ibi}
\item
  \textbf{Visites a universitats o centres tecnològics}
\item
  \textbf{Competició Programame}
\item
  \textbf{Skills Comunitat Valenciana}
\item
  Participació en fires i concursos com:
\item
  \textbf{Experimenta} (UV)
\item
  \textbf{La Navaja Negra} (ciberseguretat)
\item
  \textbf{Buca IMSEF} (projectes internacionals)
\item
  \textbf{ISIF} (fira internacional d'innovació educativa)
\end{itemize}

\hypertarget{coordinaciuxf3-i-seguiment}{%
\subsection{Coordinació i seguiment}\label{coordinaciuxf3-i-seguiment}}

L'equip docent farà una planificació anual d'activitats, tenint en
compte:

\begin{itemize}
\tightlist
\item
  Els \textbf{projectes interdisciplinars} en curs.
\item
  Les \textbf{necessitats formatives i motivacionals} de l'alumnat.
\item
  La valoració d'impacte de cursos anteriors.
\end{itemize}

Així mateix, es garantirà l'avaluació de cada activitat per mitjà de
\textbf{rúbriques, qüestionaris de satisfacció o evidències
documentals}, integrant-les com a part del desenvolupament competencial
del cicle.

\begin{center}\rule{0.5\linewidth}{0.5pt}\end{center}

\hypertarget{clip-annexos-de-suport-1}{%
\subsection{/{[}clip/{]} Annexos de
suport}\label{clip-annexos-de-suport-1}}

Els següents documents faciliten la planificació i valoració de les
activitats complementàries i extraescolars:

\begin{itemize}
\tightlist
\item
  \href{../annexos/Annex-Planificacio-Activitat/}{/{[}calendari/{]}
  Fitxa de planificació d'activitat}
\item
  \href{../annexos/Annex-Valoracio-Activitat/}{/{[}gràfic/{]} Fitxa de
  valoració d'activitat}
\end{itemize}

Es recomana utilitzar aquestes plantilles per a totes les activitats
organitzades dins del cicle FPB.

\hypertarget{criteris-per-a-lorganitzaciuxf3-del-muxf2dul-professional-de-projecte}{%
\section{18. Criteris per a l/'organització del mòdul professional de
projecte}\label{criteris-per-a-lorganitzaciuxf3-del-muxf2dul-professional-de-projecte}}

Aquest apartat \textbf{no és aplicable al cicle de Formació Professional
Bàsica d/'Informàtica d/'Oficina}, ja que aquest \textbf{no inclou el
mòdul professional de projecte} dins del seu currículum, segons el Reial
decret 127/2014 i el Reial decret 356/2014.

No obstant això, els \textbf{projectes interdisciplinaris} i activitats
col·laboratives realitzades al llarg del cicle compleixen la funció
d/'aplicació pràctica dels coneixements, d/'acord amb els principis
metodològics de l/'FPB.

\hypertarget{altres-aspectes-que-ha-de-contindre-el-pccf}{%
\section{19. Altres aspectes que ha de contindre el
PCCF}\label{altres-aspectes-que-ha-de-contindre-el-pccf}}

Aquest apartat recull \textbf{acords interns, decisions pedagògiques i
aspectes rellevants} consensuats per l'equip docent del cicle de
\textbf{Formació Professional Bàsica d'Informàtica d'Oficina}.

Tot i que no formen part d'un bloc normatiu específic, aquests aspectes
complementen el projecte curricular i \textbf{contribueixen a la cohesió
metodològica, pedagògica i organitzativa} del cicle.

\hypertarget{acords-metodoluxf2gics}{%
\subsection{1. Acords metodològics}\label{acords-metodoluxf2gics}}

\begin{itemize}
\tightlist
\item
  Ús generalitzat de metodologies actives: ABP, reptes, simulacions,
  tallers/\ldots{}
\item
  Aposta per l'\textbf{aprenentatge vivencial i contextualitzat},
  relacionat amb entorns reals.
\item
  Preferència per activitats intermodulars coordinades dins d'un mateix
  projecte o situació d'aprenentatge.
\end{itemize}

\hypertarget{acords-davaluaciuxf3}{%
\subsection{2. Acords d/'avaluació}\label{acords-davaluaciuxf3}}

\begin{itemize}
\tightlist
\item
  Ús compartit d'escales qualitatives, rúbriques i portafolis per a una
  avaluació formativa.
\item
  Disseny conjunt d'instruments d'avaluació per projectes transversals.
\item
  Establiment de mínims consensuats per a l'avaluació contínua.
\end{itemize}

\hypertarget{acords-dinclusiuxf3-i-convivuxe8ncia}{%
\subsection{3. Acords d'inclusió i
convivència}\label{acords-dinclusiuxf3-i-convivuxe8ncia}}

\begin{itemize}
\tightlist
\item
  Protocol intern per a la detecció primerenca de dificultats
  d'aprenentatge.
\item
  Coordinació mensual amb orientació per al seguiment de casos amb MREI
  o PPA.
\item
  Espais de trobada amb alumnat per tractar qüestions de convivència,
  benestar i clima d'aula.
\end{itemize}

\hypertarget{altres-acords-especuxedfics}{%
\subsection{4. Altres acords
específics}\label{altres-acords-especuxedfics}}

\begin{itemize}
\tightlist
\item
  Compromís de participació del cicle en activitats del centre (jornades
  culturals, fires, etc.)
\item
  Creació d'una comissió de projectes per coordinar accions com JustHub
  Garage o FP Skills.
\end{itemize}

\begin{quote}
🗒️ \emph{Aquest apartat pot actualitzar-se cada curs escolar amb els
nous acords establerts pel claustre docent o la comissió pedagògica del
cicle.}
\end{quote}

\hypertarget{informe-del-departament-dinformuxe0tica-i-comunicacions-sobre-la-programaciuxf3-diduxe0ctica-en-fpb}{%
\subsection{5. Informe del Departament d'Informàtica i Comunicacions
sobre la programació didàctica en
FPB}\label{informe-del-departament-dinformuxe0tica-i-comunicacions-sobre-la-programaciuxf3-diduxe0ctica-en-fpb}}

En compliment de les \textbf{Instruccions d'inici de curs 2025-2026 de
la GVA}, aquest Departament elaborarà la \textbf{programació didàctica
única de cada mòdul professional} del cicle, amb caràcter de document
marc per a tot el curs i per a tot l'alumnat, independentment del torn,
la modalitat o el règim en què s'impartisca.

No obstant això, i amb la finalitat de deixar constància en el
\textbf{Projecte Curricular de Cicle Formatiu (PCCF)}, l'Equip del
Departament d'Informàtica i Comunicacions \textbf{manifesta informe
desfavorable} davant aquesta aplicació normativa, atés que:

\begin{itemize}
\tightlist
\item
  \textbf{No s'alinea amb el canvi de paradigma educatiu} establit per
  la \textbf{LOMLOE} i la \textbf{Llei Orgànica de Formació
  Professional}, que impulsen un model competencial i
  d'\textbf{avaluació formativa i contínua}.
\item
  \textbf{Genera incoherència interetapes}, ja que en Primària, ESO i
  Batxillerat, mitjançant les Instruccions d'inici de curs, aquest canvi
  s'ha traduït en l'ús de les \textbf{Situacions d'Aprenentatge (SA)}
  com a instrument didàctic, mentre que en FP es manté un esquema basat
  en \textbf{Unitats de Programació (UP)}.
\item
  \textbf{Redueix la capacitat de connexió entre Resultats
  d'Aprenentatge, Criteris d'Avaluació i Sabers}, en no contemplar
  clarament la seua planificació i concreció en activitats
  contextualitzades.
\item
  \textbf{Allunya la FP de la seua finalitat última}, que és preparar
  l'alumnat per a entorns reals i professionals mitjançant aprenentatges
  aplicats, flexibles i basats en la resolució de reptes.
\end{itemize}

Per tant, \textbf{acatem les Instruccions vigents}, però el Departament
d'Informàtica i Comunicacions considera que la decisió \textbf{no respon
plenament a l'esperit de la legislació bàsica ni a l'orientació
pedagògica de la resta d'etapes educatives}, i recomana que en futures
adaptacions normatives es recupere la coherència global, incorporant
instruments didàctics que permeten el desenvolupament real del model
competencial (com les \textbf{Situacions d'Aprenentatge}).

\begin{center}\rule{0.5\linewidth}{0.5pt}\end{center}

\hypertarget{annex-diferuxe8ncies-entre-unitats-diduxe0ctiques-unitats-de-programaciuxf3-udup-i-situacions-daprenentatge-sa}{%
\subsection{Annex: Diferències entre Unitats Didàctiques / Unitats de
Programació (UD/UP) i Situacions d'Aprenentatge
(SA)}\label{annex-diferuxe8ncies-entre-unitats-diduxe0ctiques-unitats-de-programaciuxf3-udup-i-situacions-daprenentatge-sa}}

Les \textbf{Unitats Didàctiques (UD)} foren introduïdes amb la LOGSE i
mantingudes amb la LOMCE com a model bàsic de planificació./ Actualment,
les \textbf{Instruccions d'inici de curs de la GVA} exigeixen l'ús de
\textbf{Unitats de Programació (UP)} com a instrument de la programació
didàctica en FP.

En canvi, la \textbf{LOMLOE} i la \textbf{Llei de FP} estableixen un nou
paradigma educatiu basat en el desenvolupament competencial i
l'avaluació formativa./ Aquest canvi, en altres etapes educatives
(Primària, ESO i Batxillerat), s'ha concretat a través de les
\textbf{Situacions d'Aprenentatge (SA)}, que actuen com a instrument
didàctic central.

\begin{longtable}[]{@{}lll@{}}
\toprule
Aspecte & Unitats Didàctiques / Unitats de Programació (UD/UP) &
Situacions d'Aprenentatge (SA) \\
\midrule
\endhead
\textbf{Marc normatiu} & Introduïdes amb LOGSE i mantingudes amb LOMCE.
En FP, les Instruccions d'inici de curs 25-26 de la GVA exigeixen UP. &
La LOMLOE i la Llei de FP impulsen un model competencial i d'avaluació
formativa que en altres etapes s'ha traduït en SA. \\
\textbf{Estructura} & Seqüència tancada de continguts i activitats,
orientada a objectius preestablerts. & Organització flexible i
contextualitzada al voltant d'un repte o problema real. \\
\textbf{Focus principal} & Objectius didàctics i criteris d'avaluació
concrets, amb enfocament transmissiu. & Desenvolupament de competències
específiques i clau a través de tasques significatives. \\
\textbf{Metodologia} & Lineal, centrada en la transmissió i la repetició
de continguts. & Activa, col·laborativa i basada en la resolució de
problemes i la interdisciplinarietat. \\
\textbf{Avaluació} & Generalment sumativa, centrada en proves i
resultats individuals. & Formativa i contínua, vinculada al procés i a
l'aplicació pràctica en contextos reals. \\
\textbf{Finalitat educativa} & Adquisició ordenada de coneixements i
habilitats concretes. & Preparació integral per a la vida personal,
social i professional mitjançant contextos aplicats. \\
\bottomrule
\end{longtable}

\hypertarget{les-programacions-diduxe0ctiques-dels-muxf2duls-io-els-projectes}{%
\section{20. Les programacions didàctiques dels mòduls i/o els
projectes}\label{les-programacions-diduxe0ctiques-dels-muxf2duls-io-els-projectes}}

Les programacions didàctiques concreten com s'imparteixen els mòduls
professionals i, si escau, els projectes del cicle. S'elaboren d'acord
amb el PCCF i amb les directrius de la comissió de coordinació
pedagògica del centre, i serveixen de base per a les programacions
d'aula.

Marc normatiu

Tal com estableix l'article 9.2 del Decret 114/2025, de 29 de juliol,
del Consell,/ les programacions didàctiques han d'incloure, com a mínim,
els elements següents.

\hypertarget{concepte-i-finalitat}{%
\subsection{1. Concepte i finalitat}\label{concepte-i-finalitat}}

La programació del mòdul ha de ser un document clar, concís i útil per a
planificar l'activitat docent. Ha d'ajustar-se al que preveu el PCCF i
donar resposta a: - La seqüència i organització dels resultats
d'aprenentatge (RA) i dels criteris d'avaluació. - L'organització i
temporització dels continguts, metodologies i recursos. - Les condicions
d'impartició (presencial o semipresencial) i l'atenció a la diversitat.

La programació és única per a cada mòdul del cicle; la seua concreció
operativa es reflectirà en les programacions d'aula de cada grup.

\hypertarget{contingut-muxednim-de-la-programaciuxf3-diduxe0ctica-del-muxf2dul}{%
\subsection{2. Contingut mínim de la programació didàctica del
mòdul}\label{contingut-muxednim-de-la-programaciuxf3-diduxe0ctica-del-muxf2dul}}

\begin{itemize}
\tightlist
\item
  Dades identificatives, marc normatiu i contextualització del mòdul.
\item
  Relació entre els estàndards de competència i els mòduls del cicle
  formatiu.
\item
  Contribució dels RA a les competències generals del títol.
\item
  Esquema general i seqüenciació de les unitats de programació.
\item
  Metodologia del procés d'ensenyança-aprenentatge, incloent enfocaments
  actius i treball per projectes.
\item
  Recursos (humans, materials i digitals).
\item
  Ús d'espais i equipaments.
\item
  Mesures d'atenció a la diversitat i suports personals.
\item
  Avaluació de l'aprenentatge: instruments, moments, pesos i
  recuperació.
\item
  Activitats complementàries i extraescolars vinculades al mòdul.
\item
  Criteris i procediments per a l'avaluació del desenvolupament de la
  programació i de la pràctica docent, així com els criteris de
  qualificació.
\item
  Qualsevol altre apartat que l'equip educatiu considere rellevant dins
  del PCCF.
\end{itemize}

Recomanacions de qualitat

\begin{itemize}
\tightlist
\item
  Defineix indicadors mesurables per avaluar l'avenç dels RA.
\item
  Assegura coherència entre activitats, instruments d'avaluació i pesos.
\item
  Preveu adaptacions metodològiques i d'avaluació per a necessitats
  específiques.
\item
  Actualitza anualment la programació amb evidències
\end{itemize}

\end{document}
